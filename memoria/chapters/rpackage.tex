En este capítulo presentamos el paquete de \R desarrollado: \texttt{imbalance}, así como las tecnologías y 
metodologías empleadas en su desarrollo.

\section{¿Qué es R? ¿Por qué R?}
\citepalias{rlang} es un lenguaje pensado para estadística computacional y todo lo que ello implica: 
manipulación de datos, visualización, variedad de modelos estadísticos, etc. \R puede ser considerado una 
implementación renovada del antiguo lenguaje S, para estadística. Destaca sobretodo por su amplia gama de
paquetes, disponible en repositorios como CRAN \footnote{\url{https://cran.r-project.org/}} o 
Bioconductor \footnote{\url{https://www.bioconductor.org}}, este último orientado a bioinformática.

\R destaca por su extensibilidad, su facilidad de uso, su extensa documentación (para que un paquete pueda ser
publicado en CRAN tiene como requisito fundamental estar bien documentado), su fácil acceso a la documentación,
y su heterogénea comunidad de usuarios, siendo empleado por estadistas, bioinformáticos, científicos
de datos, informáticos o incluso investigadores asociados a medicina o biología.

\R no debe ser visto sólo como un lenguaje, sino también como la propia implementación, puesto que lo que se
asume como estándar del lenguaje es la implementación GNU-R (implementación libre), a pesar de existir otras 
como \texttt{pqR} \footnote{\url{www.pqr-project.org}} (\textit{Pretty Quick R}) o 
\texttt{FastR} \footnote{\url{https://github.com/allr/purdue-fastr}}. \R no es rápido, puesto que es un 
lenguaje interpretado y el \textit{core} de GNU-R no tiene como objetivo fundamental hacerlo más rápido, sino
proporcionar un lenguaje estable. 

La filosofía principal de la programación en R es:

\begin{itemize}
  \item \textbf{Programación funcional}: los paquetes desarrollados sólo pueden exportar funciones (técnicamente
  también ojetos \texttt{S3}, \texttt{S4} o \texttt{RC}, pero incluso los mismos trabajan con un tipo de orientación
  a objetos peculiar, donde la función recibe al objeto, y este sólo se usa para resolver sobre qué método concreto
  de los que concretan la función \textit{despacha} el objeto).
  \item \textbf{Inmutabilidad}: cualquier función del lenguaje no puede modificar un objeto que existe en
  un entorno o capa superior desde donde ha sido llamada la función (técnicamente puede hacerlo, pero con metaprogramación,
  no con evaluación convencional).
  \item \textbf{Vectorización}: \R incluye una serie de funcionales (funciones que toman otras como entrada) 
  que vectorizan operaciones (hacen que se ejecuten de manera más eficiente, aprovechando la estructura del objeto)
  sobre matrices, vectores, \textit{datasets}, listas, etc. Cuando puede usarse vectorización, es mucho más
  rápida que los bucles convencionales.
\end{itemize}

Se ha escogido \R como herramienta para desarrollar el software dada su arraigado uso en estadística y en ciencia
de datos. Aunque otros lenguajes como \texttt{Python} proporcionan herramientas tremendamente potentes para
hacer ciencia de datos, \R está menos orientado a programadores puros a mi parecer y más a obtener resultados
científicos a partir de lo programado. Se quería proporcionar una implementación de los algoritmos que 
ofreciera un balance entre eficiencia y facilidad de uso, y \R parecía la mejor opción para alcanzar dicho
objetivo. Además, la diseminación de los algoritmos principales (no únicamente en preprocesamiento de datos,
sino en ciencia de datos en general) en varios paquetes (entre los de tratatamiento de datos no balanceados cabe
citar a \citepalias{rsmote} o \citepalias{rrose} en \R) hace que no haya una fuerte dependencia de un 
único paquete de software, como ocurriría con \citepalias{scikit} en \texttt{python}.

\section{Instalación}
\subsection{Instalación de \R}
La instalación de \R en distribuciones Linux basadas en \texttt{Debian} puede efectuarse de la forma:
\begin{lstlisting}
sudo apt install r-base rbase-dev
\end{lstlisting}

En distribuciones basadas en \texttt{Arch Linux}:
\begin{lstlisting}
sudo pacman -S r
\end{lstlisting}

También es recomendable, aunque no necesario, instalar \texttt{RStudio} \footnote{\url{https://www.rstudio.com}}


La instalación del software se puede llevar a cabo usando el paquete \citepalias{rdevtools}:
\begin{lstlisting}
install.packages("devtools")
devtools::install_github("ncordon/imbalance")
\end{lstlisting}

La instalación del paquete \texttt{devtools} (primera línea) no es necesaria si ya se tiene instalado.


  