\chapter*{Introducción}

  \section*{Bibliografía empleada}
  El material principal en el que se ha basado la sección de matemáticas ha sido \citeauthor{shalev}, 
  que proporciona una detallada formalización de los fundamentos matemáticos del Aprendizaje Automático. Puntualizamos además
  una serie de referencias extras usadas en la elaboración de esta parte:
  
  \begin{itemize} 
   \item La primera parte del capítulo de introducción a teoría de la probabilidad ha sido confeccionado por el autor de 
   este trabajo a partir de \citet{caratheodory}, que ofrece una demostración guiada a través de ejercicios de dicho teorema. 
   También se ha usado como apoyo para dicho capítulo el libro \citeauthor{loeve}. 
   \item Para la parte final del primer capítulo de introducción a probabilidad se ha usado \citeauthor{shalev}, pero
   mayoritariamente el contenido de Wikipedia \citep{wiki:markov, wiki:hoeff_lemma, wiki:hoeffding}.
   \item Para el teorema \ref{th:fundamental} sobre el teorema fundamental PAC se ha usado principalmente el material de clase
   proporcionado por \citeauthor{slfetaya} para sus clases en \textit{Weizmann Institue of Science}.
  \end{itemize}
   
   Para la sección de informática se ha empleado:
   
  \begin{itemize}
   \item Para el capítulo introductorio al desbalanceo se ha empleado \citep{he2009} 
   \item Cada uno de los algoritmos implementados corresponde al material descrito en un paper de revista científica: 
   MWMOTE \citep{chawla02}, RACOG y wRACOG\citep{das2015}, RWO \citep{zhang2014}, PDFOS \citep{gao2014} y 
   NEATER \citep{almogahed2014}. Para la introducción a PDFOS se ha usado además material adaptado desde el libro\citeauthor{silverman}.
  \end{itemize}

   Para el desarrollo del software se ha seguido fundamentalmente \citep*{rhadleypkg}, que proporciona una guía detallada de
   cómo construir correctamente un paquete del lenguaje R. También se han consultado capítulos aislados de \citep*{rgillespie},
   que ofrece información muy útil sobre cómo hacer softwre eficiente en R.