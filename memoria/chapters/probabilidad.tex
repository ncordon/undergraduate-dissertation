\section{Introducción a la probabilidad}

Sea en lo que sigue $X$ un conjunto. Dados $A_i \subseteq X$, para $i=1, \ldots n$, notamos $\sum_{i=1}^n A_i$
a su unión disjunta, esto es $\bigcup_{i=1}^n A_i$ con $A_i \cap A_j = \emptyset$ cuando $i\neq j$.

\begin{definition} \textbf{Conjunto potencia}
 Definimos el conjunto potencia de $X$ a $\mathcal{P}(X):= \{A: A\subseteq X\}$.
 
 Como notación alternativa para $\mathcal{P}(X)$ usaremos $2^X$.
\end{definition}


\begin{definition} \textbf{$\sigma$-álgebra}

 $\Sigma \subseteq \mathcal{P}(X) = \{A: A\subseteq X\}$ es $\sigma$-álgebra de conjuntos sobre $X$ si se verifica:
 
 \begin{enumerate}[i]
  \item $X \in \Sigma$
  \item $\Sigma$ es cerrado para complementarios: Sea $A\in \Sigma$, entonces $A^c = X\setminus A \in \Sigma$
  \item $\Sigma$ es cerrado para uniones numerables: Sean $\{A_n\}_{n\in\mathbb{N}} \subseteq \Sigma$, entonces: 
  \[\underset{n\in\mathbb{N}}{\bigcup} A_n \in \Sigma\]
 \end{enumerate}
\end{definition}

\begin{fact}
 Sea $\Sigma$ $\sigma$-álgebra sobre $X$. Entonces:
 
 \begin{enumerate}[i]
  \item $\emptyset \in \Sigma$
  \item $\Sigma$ es cerrada para intersecciones: dados $A,B \in \Sigma$, entonces $A\cap B \in \Sigma$
  \item $\Sigma$ es cerrada para diferencias: dados $A,B \in \Sigma$, entonces $A\setminus B \in \Sigma$
 \end{enumerate}
 
 \label{fact:propsigma}
\end{fact}

\begin{proof}
 Se deducen fácilmente escribiendo $\emptyset = X^c$, $A\cap B = ((A\cap B)^c)^c = (A^c \cup B^c)^c$ y $A\setminus B = A\cap B^c$
 y usando las hipótesis de $\sigma$-álgebra.
\end{proof}



\begin{definition} \textbf{Semianillo en $X$}

 $S\subseteq \mathcal{P}(X)$ es subanillo si verifica:
 
 \begin{enumerate}[i]
  \item $\emptyset \in S$
  \item $A,B \in S$ entonces $A\cup B \in S$
  \item $A,B \in S$ entonces existen $A_1 \ldots A_n \in S$ verificándose $A\setminus B = \sum_{i=1}^n A_i$
 \end{enumerate}
\end{definition}

\begin{example}
 \begin{definition}
  Sean $X_1, X_2$ conjuntos, $\Sigma_i$ $\sigma$-álgebra sobre $A_i$. 
  
  Dados $A_1 \in \Sigma_1, A_2 \in \Sigma_2$ arbitrarios, definimos el rectángulo de lados $A_1$ y $A_2$ como:
  
  \[Rec(A_1, A_2) = A_1 \times A_2\]
 \end{definition}
 
 
 La clase de rectángulos $Rec = \{Rec(A_1, A_2): A_i \in \Sigma_i\}$ es un semianillo en $X_1 \times X_2$,
 ya que dados $R_1 = A_1 \times A_2 \in Rec, R_2 = B_1 \times B_2 \in Rec$ arbitrarios:
 
 \begin{enumerate}
  \item $\emptyset \in \Sigma_i$, y $\emptyset \times \emptyset = \emptyset$
  \item $R_1 \cap R_2 = (A_1 \cap B_1) \times (A_2 \cap B_2)$, donde $A_i \cap B_i \in \Sigma_i$ por ser 
  $\Sigma_i$ cerrada bajo intersecciones.
  \item $R_1 = A_1 \times A_2 \in Rec, R_2 = B_1 \times B_2 \in Rec$, entonces 
  $R_1 \setminus R_2 = \{(x,y): (x,y) \in A_1 \times A_2, (x,y) \notin B_1 \times B_2)\}$
  
  Es decir $R_1 \setminus R_2 = Rec(A_1\setminus B_1, A_2) \cup Rec(B_1, A_2\setminus B_2)$ y 
  $A_i \setminus B_i \in \Sigma_i$ por ser $\Sigma_i$ cerrada bajo diferencias.
 \end{enumerate}

\end{example}


\begin{definition} \textbf{Anillo en $X$}

 $R\subseteq \mathcal{P}(X)$ es anillo si verifica:
 
 \begin{enumerate}[i]
  \item $\emptyset \in R$
  \item $A,B \in R$ entonces $A\cap B \in R$
  \item $A,B \in R$ entonces $A\setminus B \in R$
 \end{enumerate}
\end{definition}


\begin{fact}
 Toda $\sigma$-álgebra es anillo. Todo anillo es semianillo.
\end{fact}

\begin{proof}
 Que $\sigma$-álgebra es más fuerte que anillo quedó probado en la proposición $\ref{fact:propsigma}$
 
 Sea ahora $R \subseteq X$ anillo y veamos que es semianillo.

 Como $A\cap B = A\setminus (A\setminus B)$, se deduce la segunda condición de la definición de semianillo.
 
 Sean $A, B \in R$, tomando $A_1 = A\setminus B \in R$, entonces se verifica la tercera condición de semianillo.
\end{proof}

\begin{counterex}
 Veamos un contraejemplo de que no todo semianillo es anillo:
 
 $\Sigma = \{\emptyset, [0,1], [1,2], [0,2]\}$ es $\sigma$-álgebra sobre $[0,2]$, y $[0,1]^2$ y $[1,2]^2$ son rectángulos
 en $[0,2]^2$, pero $[0,1]^2 \cup [1,2]^2$ no puede ser escrito como producto de $A,B \in \Sigma$.
 
 Y uno de que no todo anillo es $\sigma$-álgebra:
 
 Sea $S = \{A\subseteq \mathbb{N}: |A| < \infty\}$. Entonces se puede comprobar fácilmente que es anillo (y semianillo)
 de $\mathbb{N}$, pero no es cerrado para complementarios porque $\mathbb{N}$ no es finito.
\end{counterex}


\begin{definition}
 Sea $\mathcal{A} \subseteq \mathcal{P}(X)$. Llamamos anillo generado por $\mathcal{A}$ y lo notamos $R(A)$ al
 menor anillo que contiene a $A$:
 
 \[R(A) = \bigcap_{\begin{array}{c}R \textrm{ anillo en } X\\ A\subseteq R \end{array}} R\]
\end{definition}

Es trivial probar que la definición es correcta ($R(A)$ es anillo), ya que dados dos conjuntos $A,B \in R(A)$, 
entonces $A, B \in R$ para todo $R$ anillo contenido a $A$.

\begin{fact}
 Sea $S$ un semianillo en $X$. Entonces:
 
 \[R(S) = \{A: A=\sum_{i=1}^n A_i, n\ge 1, A_i \in S\}\]
\end{fact}

\begin{proof}
 LLamamos $\{A: A=\sum_{i=1}^n A_i, n\ge 1, A_i \in S\} = R$. Es claro que $S\subseteq R$.
 
 Sean $A = \sum_{i=1}^n A_i, B = \sum_{i=1}^k B_i \in R$ no nulos. Entonces $A\cap B = \sum_{i,j} (A_i \cap B_j)$ y $R$
 es cerrado para intersección de elementos. Además:
 
 \[A\setminus B = \cup_{i=1}^n \cap_{j=1}^p B_j^c = \cap_{j=1}^p \left(\cup_{i=1}^n (A_i \cup B_j)\right) = \cap_{j=1}^p \left(\sum_{i=1}^n (A_i \cup B_j)\right)\]
 
 Luego aplicando que $R$ es cerrado para intersecciones, llegamos también a que lo es para diferencias.
 
 Por último, como $A\cup B = A \setminus B \sum B$, $A \cup B \in R$, y como $R(S)$ debe contener por definición
 las uniones de elementos de $S$, en particular $R\subseteq R(S)$, pero hemos probado que $R$ es anillo, y $R(S)$
 es el menor anillo que contiene a $S$, luego $R(S) = R$.
\end{proof}


\begin{definition} \textbf{Medida}

 Sea $\mathcal{A} \subseteq \mathcal{P}(X)$. Llamamos medida sobre $\mathcal{A}$ a cualquier función 
 $\mu: \mathcal{A} \rightarrow [0, \infty]$ verificando:

 \begin{enumerate}[i]
  \item $\mu(\emptyset) = 0$
  \item Dados $A_n \in \mathcal{A}$ tales que $A = \sum_{i=n}{+\infty} A_n \in \mathcal{A}$, entonces 
  $\mu(A)= \sum_{i=n}{+\infty} \mu(A_n)$
 \end{enumerate}
\end{definition}

\begin{definition} \textbf{Espacio medible}

 Sea $X$ un conjunto, $\Sigma$ una $\sigma$-álgebra de conjuntos sobre $X$. A la tupla $(X,\Sigma)$ la llamamos
 espacio medible. A los elementos de $\Sigma$ los llamamos conjuntos medibles.
\end{definition}


\begin{definition} \textbf{Espacio de medida}

 Sea $(X, \Sigma)$ espacio medible, y $\mu: \Sigma \rightarrow [0,\infty]$ medida. A la tupla $(X, \Sigma, P)$ 
 la llamamos espacio de medida.
\end{definition}


\begin{definition} \textbf{Espacio de probabilidad}
 Sea $(X, \Sigma, P)$ espacio de medida. Entonces lo llamamos espacio de probabilidad si $P(\Sigma)\subseteq [0,1]$
\end{definition}


\begin{definition} \textbf{Distribución de probabilidad}

 Sea $(X, \Sigma, P)$ espacio de probabilidad. Llamamos a la tupla $\dist = (\Sigma,P)$ distribución sobre $X$. 
 Si $\dist$ es distribución sobre $X$, lo notamos $x\sim \mathcal{D}$
\end{definition}

\begin{lemma} \textbf{Desigualdad de Hoeffding}

 Sean $(X_1, \ldots X_m)$ una muestra aleatoria simple de una variable $X$, 
 $\bar{X} = \frac{1}{m} \sum_{i=1}^m X_i$ con $E[\bar{X}] = \mu$ y $P[a \le X_i \le b] = 1, i=1, \ldots m$. 
 Entonces para todo $\epsilon > 0$

 \[P\left[\left| \bar{X} - \mu \right| > \epsilon \right] \le 2e^{-2m \left(\frac{\epsilon}{b-a}\right)^2}\]
\end{lemma}