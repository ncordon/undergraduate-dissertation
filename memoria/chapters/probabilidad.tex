\section{Introducción a la probabilidad}



\begin{definition} \textbf{$\sigma$-álgebra}

 Sea $X$ un conjunto, $\Sigma \subseteq \mathcal{P}(X) = \{A: A\subseteq X\}$ es $\sigma$-álgebra de conjuntos
 de $X$ si se verifica:
 
 \begin{enumerate}[i]
  \item $X \in \Sigma$
  \item $\Sigma$ es cerrado para complementarios: Sea $A\in \Sigma$, entonces $A^c = X\setminus A \in \Sigma$
  \item $\Sigma$ es cerrado para uniones numerables: Sean $\{A_n\}_{n\in\mathbb{N}} \subseteq \Sigma$, entonces: 
  \[\underset{n\in\mathbb{N}}{\bigcup} A_n \in \Sigma\]
 \end{enumerate}
\end{definition}

\begin{definition} \textbf{Espacio medible}

 Sea $X$ un conjunto, $\Sigma$ una $\sigma$-álgebra de conjuntos sobre $X$. A la tupla $(X,\Sigma)$ la llamamos
 espacio medible. A los elementos de $\Sigma$ los llamamos conjuntos medibles.
\end{definition}


\begin{definition} \textbf{Espacio de probabilidad}

 Sea $(X, \Sigma)$ espacio medible, y $P: \Sigma \rightarrow [0,1]$ verifican los axiomas de Kolmogorov:

 \begin{enumerate}[i]
  \item $\sigma$-aditividad: dados $\{A_n\}_{n\in \mathbb{N}} \in \Sigma$ disjuntos, entonces: 
  \[P \left(\bigcup A_n\right) = \sum_{n\ge 1} P(A_n)\]
  \item $P(X) = 1$
 \end{enumerate}

 A $P$ se le llama función de probabilidad sobre $X$, y a $(X, \Sigma, P)$ espacio de probabilidad.
\end{definition}

\begin{definition} \textbf{Distribución de probabilidad}

 Sea $(X, \Sigma, P)$ espacio de probabilidad. Llamamos a la tupla $\dist = (\Sigma,P)$ distribución sobre $X$. 
 Si $\dist$ es distribución sobre $X$, lo notamos $x\sim \mathcal{D}$
\end{definition}

\begin{lemma} \textbf{Desigualdad de Hoeffding}

 Sean $(X_1, \ldots X_m)$ una muestra aleatoria simple de una variable $X$, 
 $\bar{X} = \frac{1}{m} \sum_{i=1}^m X_i$ con $E[\bar{X}] = \mu$ y $P[a \le X_i \le b] = 1, i=1, \ldots m$. 
 Entonces para todo $\epsilon > 0$

 \[P\left[\left| \bar{X} - \mu \right| > \epsilon \right] \le 2e^{-2m \left(\frac{\epsilon}{b-a}\right)^2}\]
\end{lemma}