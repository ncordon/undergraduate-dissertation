\documentclass[a4paper,11pt]{report}
\usepackage[utf8]{inputenc}
\usepackage[T1]{fontenc}
\usepackage[spanish,es-lcroman]{babel}
\usepackage{graphicx}
\usepackage{longtable}
\usepackage{float}
\usepackage{wrapfig}
\usepackage{wasysym}
\usepackage{amssymb}
\usepackage{hyperref}
%\usepackage{amsmath}
\usepackage{amsthm}
\usepackage{dsfont}

\newtheorem{theorem}{Teorema}
\newtheorem{fact}{Proposición}
\newtheorem{lemma}{Lema}
\newtheorem{corollary}{Corolario}
\newtheorem{definition}{Definición}
\setlength{\parindent}{0pt}
\setlength{\parskip}{1em}

\usepackage{color}
\newenvironment{wording}{\setlength{\parskip}{0pt}\rule{\textwidth}{0.5em}}{~\\\rule{\textwidth}{0.5em}}

\everymath{\displaystyle}

\begin{document}
% *********************************************************
% Frontmatter
% *********************************************************
\pagenumbering{roman}

\makeatletter
  \def\input@path{{./frontmatter/}}
\makeatother


\begin{titlepage}
	\centering
	\includegraphics[width=0.5\textwidth]{./imgs/ugr.png}\par
	\vspace{1cm}
        \rule{\textwidth}{0.3em}\hfill
        {\huge\bfseries 
	  Aprendizaje PAC. \par
	  Clasificación no balanceada:
	  particularización a Small Disjuncts.\par}
        \rule{\textwidth}{0.3em}\hfill
	\vspace{1cm}
        {\scshape Proyecto final de Carrera\par}
        {\scshape Ingeniería informática y Matemáticas\par}
        \vfill
        {\bf Autor \par}
	{Ignacio Cordón Castillo \par}
	{\bf Tutor \par}
	{Salvador García López \par}
	\vfill
      	\includegraphics[width=0.4\textwidth]{./imgs/by-nc-sa.png}\par
	{\large \today\par}
\end{titlepage}
\chapter*{Resumen}
  El siguiente trabajo contiene una formalización matemática del aprendizaje automático, centrándonos en clasificación binaria,
conocida como aprendizaje PAC. Para su formalización ofreceremos una introducción a $\sigma$-álgebras y $\sigma$-álgebras producto y demostraremos desigualdades
fundamentales sobre las que construiremos nuestra teoría PAC. Nuestro resultado fundamental será el teorema fundamental del
aprendizaje PAC, que relaciona estadística, combinatoria y automático. También describiremos otro paradigma de aprendizaje,
más laxo que la cognoscibilidad PAC: el aprendizaje no-uniforme.

El desarrollo informático presenta el problema de la clasificación no balanceada. Presentaremos distintas aproximaciones 
a la resolución del problema, centrándonos en una, el \textit{oversampling}. Dentro del \textit{oversampling} describiremos
las ideas subyacentes en una serie de algoritmos recientes, aún no disponibles en \R, y los implementaremos en dicho 
lenguaje, apoyándonos en \texttt{C++} para acelerar algunos de ellos. Por último, se hará una pequeña experimentación
con \textit{small disjuncts}, un concepto estrechamente ligado al debalanceo de clases.

\paragraph{Palabras clave}
Aprendizaje Automático, aprendizaje PAC, clasificación no balanceada, \textit{oversampling}, \textit{small disjuncts}
\chapter*{Abstract}
  Machine learning foundations and algorithms have been one of the most studied areas in both mathematics and computer
science in the past few years, due to the constant improvement in computer features. Fundamental questions arises: 
what does it mean to extract knowledge out of data?, when can we learn from the data?, how much data do we need?

One of the most researched topics in machine learning is classification problems: given a sequence of elements belonging to
a domain, which have been labeled, how could we learn from them in a way that if new samples arrive we would be able to 
label them according to the knowledge acquired?. We will specially focus on binary classification problems, those which 
only have two possible labels. We will also study, in a more empirical framework, the particular case in which one 
class has more examples than the other one: imbalance classification.

The aim of this work is both to provide mathematical foundations for machine learning, which has in the PAC learning 
an excellent formalization, and to study and code some of the most recent algorithms that have been published in scientific
journals on the topic of imbalance classification.

\section*{Mathematical introduction}
We introduce concepts suchs as product ring and semiring of sets, and their relationship with $\sigma$-algebras. We give
the notion of measure, without distinction between $\sigma$-algebras and other sets, whereas in classic literature the latter
ones are called premeasures. We also introduce the outer measures, and we advance towards the construction of a $\sigma$-algebra
based on a premeasure over some space, and the extension of that outer measure to a measure. The result which gives
such extension is Carathéodory's theorem.

This introduction's purpose is to provide us with basic tools to develop the following theory and to answer the question:
given a set of probabilistic spaces, does it exist a product $\sigma$-algebra space in which the product of the sets has
the product of the probabilities as measure?.

This very first part includes demonstrations for Markov and Hoeffding's inequalities, the former one based on the Hoeffding
lemma, which will be key inequalities in our subsequent progress.

\section*{Machine learning introduction and PAC framework}
We will provide motivations for the machine learning theory, as well as basic definitions such as domain set, label set, 
true labeling, instance generation, training set, hypothesis, hypothesis' error or learning algorithm. The error of an
hypothesis respect to the true labeling one will be defined as the expectation that the two are different. The error over
a training set, as the mean of the number of instances in which it fails.

We will also describe a relation between hypothesis and the set in which those hypothesis have value $1$. This will allow 
as to conveniently work both with hypothesis or sets. The star algorithms we will be using will be empirical risk 
minimizers, that is, all those which try to minimize the error over a given training set.

All those concepts will sum up to the very first concept of PAC learning: we will say that a hypothesis collection is 
learnable if we can allways guarantee an small error from one of them with respect to the true labeling function, under
certain restriction of probabilistic confidence. In particular, we will prove that every finite collection of hypothesis
is learnable, and that there are infinite learnable collections, such as rectangles hypothesis.

The PAC notions will be weakened as much as possible until we arrive to a more flexible definition of learning: APAC 
learning, which is rather similar to PAC learning, with the main difference that we can define our own error function,
and we allow that instances of both classes to suffer from overlaping (that is, an instance could belong to both classes). As
a result, we will show that the concept of APAC implies the PAC concept under certain conditions.


\section*{Uniform learning and further work}

We will define Glivenko Cantelli classes (uniform classes) of hypothesis, and show that being Glivenko Cantelli is an
stronger assumption than being APAC. As a result of the work developed under this chapter, we will prove again that finite
classes as learnable, but under a random error function and with worse asymptotic complexity.

Vapnik Chervonenkis classes will also be introduced, along with the concept of shattering. This is a pure combinatorics
definition. The Sauer-Shelah lemma will gives us a relationship between the Vapnik Chervonenkis dimension and the number
of possible hypothesis in terms of their image of a training set of fixed size. 

At this stage, we will have all the pieces to prove the fundamental theorem of PAC learning, that connects statistical
concepts, such as Glivenko Cantelli classes, combinatorial ones (the Vapnik Chervonenkis dimension) and PAC and APAC
concepts.

\section*{Introduction to the imbalance problem}

In order to study some machine learning algorithms, we will weaken our concept of learning until we reach a framework in which
we can easily explain one of data science most studied problems: imbalanced classification. We will describe the suitable
approaches to solve that problem: undersampling and oversampling with and without filtering of instances and cost-sensitive 
framework. We will also provide a set of measurements of the quality of an algorithm, since most of nowadays algorithms
tend to understimate the importance of the minority examples.


Our main focus will be on oversampling, the creation of synthetic samples belonging to the minority class. We will provide
a description of a classic algorithm: SMOTE. Although it was first described 15 years ago, it is still a reference in the
state of the art, and a lot of other algorithms have arised by means of modification of the former one. It will be the case
of our MWMOTE algorithm.

\section*{Description of the coded algorithms}

We will analyze a series of algorithms that are not currently present in any package of \R programming language. A brief 
background theory for those algorithms will be provided. We will start with MWMOTE algorithm, which tries to fix some of
the issues of SMOTE algorithm, such as creation of new instances using noisy ones, or producing instances between two
different minoriry class clusters.

RACOG will also be addresed, as an algorithm that approximates discrete distributions using marginal pairwise ones and 
information theory. Once the distribution has been approximated, it will extract samples following a Monte Carlo scheme
called Gibbs Sampling. wRACOG will be a modification of RACOG that aims to improve a single classifier provided as parameter.

The next algorithm we will code is RWO. RWO finds its foundation on a statistics theorem that we will prove, that guarantees
us that under asymptotic assumptions and generation scheme, our newly generated instances will preserve the original mean
and variance of the training data set.

PDFOS will be another of the algorithms selected. It is based on multivariate Gaussian kernel density estimations. We will
provide an introduction to the kernel density estimators pointing out their relationship with histograms and we will give
a measure of the error of the estimation, the Mean Integrated Squared Error. PDFOS will try to adapt the bandwith parameter
for a sum of multivariate Gaussian kernels, that is, a multiplicative constant for the unbiased covariance of the minority
training samples. A gradient descendent method will be used to optimize such parameter.

The last algorithm we are going to study is NEATER, a filtering algorithm to clean up noisy instances created as a result of
oversampling. NEATER is based on game theory and Nash profile equilibriums, that guarantee that if an instance could play
two possible roles, it will tend to stabilize as one of them, giving a series of payoffs for itself and its neighbour 
instances.

\section*{Developed software}
An implementation for those algorithms will be made using the \R programming language combined with \texttt{C++} chunks, and 
using \texttt{Armadillo} library that provides vectorized operations under \texttt{C++}. \R language provides a trade-off 
between ease of use for the users and speed of the code, the second one will be used to speed up our algorithms when 
possible. As a result, we will develop a package called \texttt{imbalance} to fill the lack of implementation of the 
described algorithms in \R.

It is important to note not only the code, but the methodology used to develop the code: under a GPLv2 or later license,
allocated in a public Github repo and with continuous integration to provide unit testing and periodic updates of
online web documentation page.

\section*{Experiments on small disjuncts}
Finally, we will conduct a simple experiment on small disjuncts problem, which arise as a direct result of class imbalance
problem. The experiment will somehow show that when we overcome the imbalance of the datasets, small disjuncts tend to
vanish.

We propose a simple measure of the small disjuncts which a dataset has: given a C4.5 tree that labels a set of examples, we 
will consider a small disjunct all those leaves which have less than a given threshold examples. We will also measure the
mean size of the leaves in terms of examples classified, sometimes called coverage.

\paragraph{Keywords}
Machine learning, PAC learning, imbalanced classification, \textit{oversampling}, \textit{small disjuncts}

\tableofcontents

\break 

% *********************************************************
% Mainmatter
% *********************************************************
\pagenumbering{arabic}
\setcounter{page}{1}

\makeatletter
  \def\input@path{{./chapters/}}
\makeatother


\chapter*{Introducción}
  
\chapter*{Objetivos}
  
  
\part{Matemáticas}
\chapter{Introducción}
  Basado mayormente en el contenido de \cite{shwartz_understanding_ml}
  
  

\section{Motivación}

El objetivo del aprendizaje automático es convertir datos en conocimiento a través de un razonamiento inductivo, de manera que
proporcionándole datos a una máquina, seamos capaces de extraer un conocimiento (una generalización de los datos, que nos permita
inferir información a partir de nuevos datos). Surge la pregunta de por qué es necesario el aprendizaje automático o 
\textit{machine learning}, si la estadística también se encarga de obtener conocimiento a partir de unos datos.

\subsection{¿Por qué necesitamos \textit{machine learning}?}
\begin{enumerate}[i]
 \item Para resolver \textbf{tareas que requieren automatización}: entran dentro de esta categoría tanto aquellas tareas para
 las que no existe una axiomatización o un conocimiento exacto, como pueden ser el reconocimiento de dígitos o de voz, como 
 aquellas tareas que requieren del análisis de un gran número de datos, y quedan fuera de la capacidad humana para realizar
 un análisis estadístico manual por ejemplo. En el primer caso necesitamos apoyarnos en conocimiento auxiliar (por ejemplo, 
 un conjunto de dígitos o de muestras de voz preetiquetadas con los que poder comparar muestras sin etiquetar/clasificar); 
 en el segundo, se hace necsario el uso de una máquina para poder extraer conocimiento de todos los datos.
 
 \item \textbf{Tareas que requieren adaptatividad}: como puede ser el reconocimiento de texto, donde no hay un patrón único de
 escritura para cada persona, y necesitamos que el algoritmo pueda generalizar eso.
\end{enumerate}

\subsection{Áreas relacionadas con el aprendizaje}
Entre las áreas relacionadas con el aprendizaje automático, cabe mencionar:

\begin{enumerate}[i]
 \item \textbf{Inteligencia Artificial}
 \item \textbf{Algorítmica}: debemos analizar el tiempo asintótico de los algoritmos mediante los que aprende la máquina.
 \item \textbf{Inferencia}: entre las diferencias que podemos mencionar con la estadística convencional, destaca la necesidad de 
 programar las tareas, dado el volumen de datos con el que normalmente se trabaja, mientras que en muchos análisis estadísticos basta 
 lápiz y papel. También destaca la \textbf{independencia respecto a distribución} con la que se trabaja (no se asume una distribución
 determinada sobre los datos). La principal diferencia del aprendizaje automático respecto a la inferencia es que la inferencia
 se encarga de comprobar la validez de las hipótesis que propone el estadista, mientras que el algoritmo de 
 \textit{machine learning} genera hipótesis para unos datos determinados, con unas ciertas condiciones de aproximación y error.
 \item \textbf{Álgebra lineal}
 \item \textbf{Optimización de algoritmos}
\end{enumerate}

\subsection{Ejemplo práctico}\label{sec:first-ex}
Pensemos en un ejemplo: tenemos clientes de un banco que quieren solicitar un préstamo, y a todos se les categoriza
el nivel de ingresos y el tamaño del préstamo (cómo de grande es su importe). Estas variables se miden 
en una escala de $0$ a $1$ donde $1$ es el nivel máximo. Queremos etiquetar a cada cliente como $S$:conceder préstamo o 
$N$:no conceder préstamo. Identificamos $0\equiv N$ y $1\equiv S$.

Dado un histórico de $m\in \mathbb{N}$ clientes a los que se les concedieron y devolvieron o no préstamos, tenemos una tupla 
$((x_1, y_1), \ldots (x_m, y_m))$ donde $x_i = ((x_i)_1, (x_i)_2) \in [0,1]^2$ y $y_i \in \{0,1\}$, y $(x_i)_1$ representa el nivel 
ingresos mensuales, y $(x_i)_2$ el tamaño del préstamo solicitado. Tenemos a los clientes clasificados en función de si 
devolvieron los préstamos o no.

Quiero encontrar una función que me ofrezca una predicción sobre cualquier cliente, para minimizar posibles pérdidas del banco, es
decir, busco $f:[0,1]^2 \rightarrow \{0,1\}$, que llamaremos predicción.

Asumimos que los datos de los clientes de que disponemos no van a tener una distribución determinada y
van a ser idéntica e independientemente distribuidos. El histórico siempre va a crecer, y no sabemos cómo va a hacerlo, queriendo aprovechar
al máximo la información del mismo. Es decir, los datos de un cliente provienen de una variable aleatoria que sigue una determinada
distribución $X \sim \mathcal{D}$; y tenemos muestas de clientes (que llamaremos conjunto de entrenamiento), variable i.i.d
$S \sim \mathcal{D}^m$. 

También asumiremos que tenemos una clase de predicciones $\mathcal{H} \subseteq \{0,1\}^{[0,1]^2}$, de entre las que existe
una óptima. Asumiremos por simplicidad que la clase de predicciones son subrectángulos de $[0,1]^2$, esto es, funciones:

\[h_{a,b,c,d} = \mathds{1}_{[a,b]\times[c,d]}, \qquad [a,b]\times [c,d] \subseteq [0,1]^2\]

\img{./imgs/rect-ex.png}{0.85}

También necesitaremos definir una medida de acierto: ¿cómo de buena es una predicción?.

\section{Modelo matemático}

Podemos dar unas notaciones/definiciones básicas que utilizaremos de aquí en adelante, en base a lo descrito en \ref{sec:first-ex}:

\begin{itemize}
\item \textbf{Dominio}: $\mathcal{X}$. Llamamos una instancia a $x\in \mathcal{X}$
\item \textbf{Conjunto de etiquetas}: $\mathcal{Y}$ consideramos $\{0,1\}$, lo que nos restringe al paradigma binario. En ocasiones 
también usaremos $\mathcal{Y} = \{-1,1\}$ para las etiquetas.
\item \textbf{Verdadero etiquetado}: \sloppy Asumimos la existencia de una función ${f: \mathcal{X} \rightarrow \mathcal{Y}}$ 
que devuelve el verdadero etiquetado de todas las instancias.
\item \textbf{Generación de instancias}: \fussy Asumimos la existencia de una distribución de probabilidad $\mathcal{D} = (\mathcal{B}, P)$, 
con $x\sim \mathcal{D}$, donde $A\subseteq \mathcal{P}(\mathcal{X})$ es $\sigma$ álgebra de conjuntos sobre $\mathcal{X}$ y $P: \mathcal{B} \rightarrow [0,1]$
función de probabilidad. La distribución de probabilidad nos da información sobre la probabilidad de extraer cada posible instancia desde 
$x \in \mathcal{X}$. Por evitar conflictos, usualmente escribiremos $P$ como $P_{x\sim \mathcal{D}}$.

\item \textbf{Conjunto de entrenamiento}: $S = ((x_1,y_1) \ldots (x_m,y_m)) \in (\mathcal{X} \times \mathcal{Y})^m$ 
Nótese que llamarlo conjunto puede dar lugar a confusión, puesto que se trata de una tupla. Notaremos $S_x = (x_1, \ldots x_m)$

De momento asumiremos que las etiquetas del conjunto de entrenamiento se corresponden con el verdadero etiquetado: 
$y_i = f(x_i)$, por lo que no podemos tener dos instancias con etiquetas diferentes.

La elección de $S_x$ es idéntica e independientemente distribuida, esto es $x_i \sim \mathcal{D}$ para todo $i=1, \ldots, m$.
Lo notamos $S_x \sim \mathcal{D}^m$

\item \textbf{Hipótesis/clasificador/predicción}: cada posible aplicación perteneciente a 
$\{h: h:\mathcal{X} \rightarrow \mathcal{Y}\} := 2^{\mathcal{X}}$. 

\item \textbf{Algoritmo de aprendizaje}: Llamamos algoritmo de aprendizaje a cualquier aplicación que tome conjuntos de entrenamiento
y devuelva hipótesis sobre el problema:

\[A: \underset{m\in \mathbb{N}}{\bigcup} (\mathcal{X}\times\mathcal{Y})^m \rightarrow 2^{\mathcal{X}}\]

Asumimos que el algoritmo no tiene acceso a la función de verdadero etiquetado $f: \mathcal{X} \rightarrow \mathcal{Y}$ ni a
la distribución $\mathcal{D}$.

\item \textbf{Error del clasificador}: Definimos el error del clasificador, suponiendo 
$\{x\in \mathcal{X} : h(x) \neq f(x)\} := [h\neq f] \in \mathcal{B}$ como:

\[L_{D,f}(h) :=  P_{x\sim \mathcal{D}} [h \neq f]\]
\end{itemize}


\begin{enumerate}
\item Minimización del riesgo empírico (ERM)
\label{sec-3-4-1-1}

\begin{definition}
\textbf{Riesgo empírico (ER)}

Definimos el riesgo empírico o error empírico como:

\[L_S(h) = \frac{|i\in {1\ldots m}: h(x_i) \neq y_i|}{m}\]
\end{definition}

Podemos pensar en él como el error del clasificador sobre el conjunto de entrenamiento. El paradigma que intenta buscar una hipótesis que minimice el error empírico recibe el nombre de \emph{Minimización de Riesgo Empírico - ERM} y notamos $ERM(S)$ al clasificador que obtenemos basándonos en este paradigma para un determinado conjunto de entrenamiento $S$.

Este error no es siempre óptimo. Pensemos en el siguiente ejemplo:

Sea $\mathcal{X} = \mathbb{R}$, $\mathcal{D}$ la distribución uniforme sobre $[0,2]\subset \mathbb{R}$, y la siguiente función:

\[f(x) = \left\{\begin{array}{lcl}
1 && x\in [0,1]\\
0 && x\in \mathbb{R}\setminus [0,1]
\end{array}\right.\]


$S = \{(x_1,y_1), \ldots (x_m, y_m)\}$ un conjunto de entrenamiento de tamaño $m$ sin elementos repetidos y el clasificador:

\[h_S(x) = \left\{\begin{array}{lcl}
y_i && \exists i\in \{1\ldots m\} : x=x_i\\
0 && \nexists i\in \{1\ldots m\} : x=x_i
\end{array}\right.\]

Este clasificador es perfecto respecto a la minimización de riesgo empírico, pero $\mathbb{P}_{x\sim \mathcal{D}}[h_S(x)] = 1/2$, es decir, tiene el mismo nivel de acierto que el clasificador idénticamente 1. A este fenómeno lo denominamos \textbf{overfitting}.

\item ERM con \emph{sesgo inductivo}
\label{sec-3-4-1-2}

Se intenta corregir el ERM corrigiendo el espacio de búsqueda, esto es, la clase de hipótesis $\mathcal{H}$ desde la que el algoritmo puede escoger un $h: \mathcal{X}\rightarrow \mathcal{Y}$. Llamamos a esto \emph{sesgo inductivo} puesto que se asumirá una determinada clase de funciones $\mathcal{H}$ en función de las características del problema.

Notaremos a este nuevo paradigma $ERM_{\mathcal{H}}(S)$, y lo definimos de manera que:

\[ERM_{\mathcal{H}}(S) := h_S \in argmin_{h\in \mathcal{H}} L_S(h)\]

Definimos la propiedad de factibilidad, que usaremos más adelante.

\begin{definition}
\textbf{Propiedad de factibilidad}

Existe  $\bar{h} \in \mathcal{H}$ verificando $L_{D,f}(\bar{h}) = 0$.
\end{definition}

La hipótesis de factibilidad implica que $\mathbb{P}_{S\sim \mathcal{D}^m}[L_S(\bar{h})=0] = 1$, y por tanto $\mathbb{P}_{S\sim \mathcal{D}^m}[L_S(h_S)=0]=1$.

El valor $L_{\mathcal{D},f}(h_S)$ dependerá del conjunto de entrenamiento $S$, y la elección del mismo está sometida al azar. Además, necesitamos definir cómo de buena será la predicción.
\end{enumerate}

\part{Informática}

\bibliography{references}
\bibliographystyle{IEEEtran}

\end{document}