\documentclass[a4paper,11pt]{report}
% Change this line for classicthesis
%\documentclass[dottedtoc, headinclude, footinclude=true]{scrreprt}

\makeatletter
  \def\input@path{{./frontmatter/}}
\makeatother

% *************************************************************************************************************
% Importa archivo de configuración
% *************************************************************************************************************
\usepackage[utf8]{inputenc}
\usepackage[T1]{fontenc}
\usepackage[spanish,es-lcroman]{babel}
\usepackage{graphicx}
\usepackage{longtable}
\usepackage{float}
\usepackage{wrapfig}
\usepackage{amssymb}
\usepackage{hyperref}
\usepackage{amsmath}
\usepackage{amsthm}
\usepackage{wasysym}
\usepackage{dsfont}
\usepackage{enumerate}
% Change this line for classicthesis
% \usepackage[pdfspacing]{classicthesis}

% *************************************************************************************************************
% Entorno para ejemplos y contraejemplos
% https://tex.stackexchange.com/questions/5223/command-for-argmin-or-argmax
% *************************************************************************************************************
\usepackage{mdframed} % Add easy frames to paragraphs
\usepackage{lipsum} % For dummy text
\usepackage{xcolor}
\usepackage{xparse} % Add support for \NewDocumentEnvironment
\definecolor{graylight}{cmyk}{.30,0,0,.67} % define color using xcolor syntax

\newmdenv[ % Define mdframe settings and store as leftrule
  linecolor=graylight,
  topline=false,
  bottomline=false,
  rightline=false,
  skipabove=\topsep,
  skipbelow=\topsep
]{leftrule}

\NewDocumentEnvironment{example}{O{\textbf{Ejemplo:}}}
{\begin{leftrule}\noindent\textcolor{black}{#1}\par}
{\end{leftrule}}

\NewDocumentEnvironment{counterex}{O{\textbf{Contraejemplo:}}}
{\begin{leftrule}\noindent\textcolor{black}{#1}\par}
{\end{leftrule}}

% *************************************************************************************************************
% Entorno para demostraciones dentro de demostraciones
% *************************************************************************************************************
\newenvironment{subenv}
{\setlength\leftskip{-2em}
\setlength\leftskip{2em}}

% *************************************************************************************************************
% Definiciones de teoremas, corolarios, lemas
% *************************************************************************************************************
\newtheorem{thm}{theorem}[section]
\newtheorem{theorem}[thm]{Teorema}
\newtheorem*{theorem*}{Teorema}
\newtheorem{fact}[thm]{Proposición}
\newtheorem*{fact*}{Proposición}
\newtheorem{lemma}[thm]{Lema}
\newtheorem*{lemma*}{Lema}
\newtheorem{corollary}[thm]{Corolario}
\newtheorem*{corollary*}{Corolario}
\newtheorem{definition}[thm]{Definición}
\newtheorem*{definition*}{Definición}


% *************************************************************************************************************
% Comandos
% *************************************************************************************************************

% Inserción de imágenes
\newcommand{\img}[2]{
  \begin{center}
  \includegraphics[width=#2\textwidth]{#1}
  \end{center}
}

\newcommand{\imgcaption}[3]{
\begin{figure}[H]
  \begin{center}
  \includegraphics[width=#2\textwidth]{#1}
  \end{center}
  
  \caption{#3}
 \end{figure}
}

% arg min
\DeclareMathOperator*{\argmin}{arg\,\min }
\DeclareMathOperator*{\argmax}{arg\,\max }

% norma || ||
\newcommand{\norm}[1]{||{#1}||}


% Para hacer el underbrace de columnas de matrices
% Fuente: https://tex.stackexchange.com/questions/102460/underbraces-in-matrix-divided-in-blocks

\newcommand\undermat[2]{%
  \makebox[0pt][c]{$\smash{\underbrace{\phantom{%
    \begin{array}{r} #2 \end{array}}}_{\text{$#1$}}}$}#2}


% Abreviaciones para escribir menos
\newcommand{\mprob}{\underset{S\sim \mathcal{D}^m}{P}}
\newcommand{\dosmprob}{\underset{S\sim \mathcal{D}^{2m}}{P}}
\newcommand{\munoprob}{\underset{S_1\sim \mathcal{D}^m}{P}}
\newcommand{\mdosprob}{\underset{S_2\sim \mathcal{D}^m}{P}}
\newcommand{\mmprob}{\underset{\begin{subarray}{c} 
				S_1 \sim \mathcal{D}^{m} \\ 
				S_2 \sim\mathcal{D}^m 
				\end{subarray}}{P}}
\newcommand{\prob}{\underset{x\sim \mathcal{D}}{P}}
\newcommand{\zprob}{\underset{z\sim \mathcal{D}}{P}}
\newcommand{\dist}{\mathcal{D}}
\newcommand{\expect}{\mathbb{E}}
\newcommand{\mexpecti}[1]{\underset{S\sim \mathcal{D}_{#1}^m}{\mathbb{E}}}
\newcommand{\mexpect}{\underset{S\sim \mathcal{D}^m}{\mathbb{E}}}
\newcommand{\dosmexpect}{\underset{S\sim \mathcal{D}^{2m}}{\mathbb{E}}}
\newcommand{\ppos}{P^{+}}
\newcommand{\pneg}{P^{-}}
\newcommand{\npos}{N^{+}}
\newcommand{\nneg}{N^{-}}
\newcommand{\spos}{S^{+}}
\newcommand{\sneg}{S^{-}}

% *************************************************************************************************************
% Pseudocódigo de algoritmos
% *************************************************************************************************************
\usepackage{algorithm}
\usepackage{algorithmic}
\floatname{algorithm}{Algoritmo}
\renewcommand{\listalgorithmname}{Lista de algoritmos}
\renewcommand{\algorithmicrequire}{\textbf{Entrada:}}
\renewcommand{\algorithmicensure}{\textbf{Salida:}}
%\renewcommand{\algorithmicfunction}{Función}
%\renewcommand{\algorithmicend}{\textbf{fin}}
%\renewcommand{\algorithmicif}{\textbf{si}}
%\renewcommand{\algorithmicthen}{\textbf{entonces}}
%\renewcommand{\algorithmicelse}{\textbf{en otro caso}}
%\renewcommand{\algorithmicelsif}{\algorithmicelse,\ \algorithmicif}
%\renewcommand{\algorithmicendif}{\algorithmicend\ \algorithmicif}
%\renewcommand{\algorithmicfor}{\textbf{para }}
%\renewcommand{\algorithmicforall}{\textbf{para cada}}
%\renewcommand{\algorithmicdo}{\textbf{}}
%\renewcommand{\algorithmicendfor}{\algorithmicend\ \algorithmicfor}
%\renewcommand{\algorithmicwhile}{\textbf{mientras}}
%\renewcommand{\algorithmicendwhile}{\algorithmicend\ \algorithmicwhile}
%\renewcommand{\algorithmicloop}{\textbf{repetir}}
%\renewcommand{\algorithmicendloop}{\algorithmicend\ \algorithmicloop}
%%\renewcommand{\algorithmicrepeat}{\textbf{repetir}}
%\renewcommand{\algorithmicuntil}{\textbf{hasta que}}
%\renewcommand{\algorithmicprint}{\textbf{imprimir}} 
%\renewcommand{\algorithmicreturn}{\textbf{devolver}} 
%\renewcommand{\algorithmictrue}{\textbf{true }} 
%\renewcommand{\algorithmicfalse}{\textbf{false }} 
%\renewcommand{\algorithmicand}{\textbf{y}}
%\renewcommand{\algorithmicor}{\textbf{o}}


\makeatletter
\def\NEWLINE{\STATE{}}
\makeatother

% *************************************************************************************************************
% Otras configuraciones
% *************************************************************************************************************
\setlength{\parindent}{0pt}
\setlength{\parskip}{1em}
\everymath{\displaystyle}



\begin{document}
% *************************************************************************************************************
% Frontmatter
% *************************************************************************************************************
\pagenumbering{roman}


\begin{titlepage}
	\centering
	\includegraphics[width=0.5\textwidth]{./imgs/ugr.png}\par
	\vspace{1cm}
        \rule{\textwidth}{0.3em}\hfill
        {\huge\bfseries 
	  Aprendizaje PAC. \par
	  Clasificación no balanceada:
	  particularización a Small Disjuncts.\par}
        \rule{\textwidth}{0.3em}\hfill
	\vspace{1cm}
        {\scshape Proyecto final de Carrera\par}
        {\scshape Ingeniería informática y Matemáticas\par}
        \vfill
        {\bf Autor \par}
	{Ignacio Cordón Castillo \par}
	{\bf Tutor \par}
	{Salvador García López \par}
	\vfill
      	\includegraphics[width=0.4\textwidth]{./imgs/by-nc-sa.png}\par
	{\large \today\par}
\end{titlepage}
\chapter*{Resumen}
  El siguiente trabajo contiene una formalización matemática del aprendizaje automático, centrándonos en clasificación binaria,
conocida como aprendizaje PAC. Para su formalización ofreceremos una introducción a $\sigma$-álgebras y $\sigma$-álgebras producto y demostraremos desigualdades
fundamentales sobre las que construiremos nuestra teoría PAC. Nuestro resultado fundamental será el teorema fundamental del
aprendizaje PAC, que relaciona estadística, combinatoria y automático. También describiremos otro paradigma de aprendizaje,
más laxo que la cognoscibilidad PAC: el aprendizaje no-uniforme.

El desarrollo informático presenta el problema de la clasificación no balanceada. Presentaremos distintas aproximaciones 
a la resolución del problema, centrándonos en una, el \textit{oversampling}. Dentro del \textit{oversampling} describiremos
las ideas subyacentes en una serie de algoritmos recientes, aún no disponibles en \R, y los implementaremos en dicho 
lenguaje, apoyándonos en \texttt{C++} para acelerar algunos de ellos. Por último, se hará una pequeña experimentación
con \textit{small disjuncts}, un concepto estrechamente ligado al debalanceo de clases.

\paragraph{Palabras clave}
Aprendizaje Automático, aprendizaje PAC, clasificación no balanceada, \textit{oversampling}, \textit{small disjuncts}
\chapter*{Abstract}
  Machine learning foundations and algorithms have been one of the most studied areas in both mathematics and computer
science in the past few years, due to the constant improvement in computer features. Fundamental questions arises: 
what does it mean to extract knowledge out of data?, when can we learn from the data?, how much data do we need?

One of the most researched topics in machine learning is classification problems: given a sequence of elements belonging to
a domain, which have been labeled, how could we learn from them in a way that if new samples arrive we would be able to 
label them according to the knowledge acquired?. We will specially focus on binary classification problems, those which 
only have two possible labels. We will also study, in a more empirical framework, the particular case in which one 
class has more examples than the other one: imbalance classification.

The aim of this work is both to provide mathematical foundations for machine learning, which has in the PAC learning 
an excellent formalization, and to study and code some of the most recent algorithms that have been published in scientific
journals on the topic of imbalance classification.

\section*{Mathematical introduction}
We introduce concepts suchs as product ring and semiring of sets, and their relationship with $\sigma$-algebras. We give
the notion of measure, without distinction between $\sigma$-algebras and other sets, whereas in classic literature the latter
ones are called premeasures. We also introduce the outer measures, and we advance towards the construction of a $\sigma$-algebra
based on a premeasure over some space, and the extension of that outer measure to a measure. The result which gives
such extension is Carathéodory's theorem.

This introduction's purpose is to provide us with basic tools to develop the following theory and to answer the question:
given a set of probabilistic spaces, does it exist a product $\sigma$-algebra space in which the product of the sets has
the product of the probabilities as measure?.

This very first part includes demonstrations for Markov and Hoeffding's inequalities, the former one based on the Hoeffding
lemma, which will be key inequalities in our subsequent progress.

\section*{Machine learning introduction and PAC framework}
We will provide motivations for the machine learning theory, as well as basic definitions such as domain set, label set, 
true labeling, instance generation, training set, hypothesis, hypothesis' error or learning algorithm. The error of an
hypothesis respect to the true labeling one will be defined as the expectation that the two are different. The error over
a training set, as the mean of the number of instances in which it fails.

We will also describe a relation between hypothesis and the set in which those hypothesis have value $1$. This will allow 
as to conveniently work both with hypothesis or sets. The star algorithms we will be using will be empirical risk 
minimizers, that is, all those which try to minimize the error over a given training set.

All those concepts will sum up to the very first concept of PAC learning: we will say that a hypothesis collection is 
learnable if we can allways guarantee an small error from one of them with respect to the true labeling function, under
certain restriction of probabilistic confidence. In particular, we will prove that every finite collection of hypothesis
is learnable, and that there are infinite learnable collections, such as rectangles hypothesis.

The PAC notions will be weakened as much as possible until we arrive to a more flexible definition of learning: APAC 
learning, which is rather similar to PAC learning, with the main difference that we can define our own error function,
and we allow that instances of both classes to suffer from overlaping (that is, an instance could belong to both classes). As
a result, we will show that the concept of APAC implies the PAC concept under certain conditions.


\section*{Uniform learning and further work}

We will define Glivenko Cantelli classes (uniform classes) of hypothesis, and show that being Glivenko Cantelli is an
stronger assumption than being APAC. As a result of the work developed under this chapter, we will prove again that finite
classes as learnable, but under a random error function and with worse asymptotic complexity.

Vapnik Chervonenkis classes will also be introduced, along with the concept of shattering. This is a pure combinatorics
definition. The Sauer-Shelah lemma will gives us a relationship between the Vapnik Chervonenkis dimension and the number
of possible hypothesis in terms of their image of a training set of fixed size. 

At this stage, we will have all the pieces to prove the fundamental theorem of PAC learning, that connects statistical
concepts, such as Glivenko Cantelli classes, combinatorial ones (the Vapnik Chervonenkis dimension) and PAC and APAC
concepts.

\section*{Introduction to the imbalance problem}

In order to study some machine learning algorithms, we will weaken our concept of learning until we reach a framework in which
we can easily explain one of data science most studied problems: imbalanced classification. We will describe the suitable
approaches to solve that problem: undersampling and oversampling with and without filtering of instances and cost-sensitive 
framework. We will also provide a set of measurements of the quality of an algorithm, since most of nowadays algorithms
tend to understimate the importance of the minority examples.


Our main focus will be on oversampling, the creation of synthetic samples belonging to the minority class. We will provide
a description of a classic algorithm: SMOTE. Although it was first described 15 years ago, it is still a reference in the
state of the art, and a lot of other algorithms have arised by means of modification of the former one. It will be the case
of our MWMOTE algorithm.

\section*{Description of the coded algorithms}

We will analyze a series of algorithms that are not currently present in any package of \R programming language. A brief 
background theory for those algorithms will be provided. We will start with MWMOTE algorithm, which tries to fix some of
the issues of SMOTE algorithm, such as creation of new instances using noisy ones, or producing instances between two
different minoriry class clusters.

RACOG will also be addresed, as an algorithm that approximates discrete distributions using marginal pairwise ones and 
information theory. Once the distribution has been approximated, it will extract samples following a Monte Carlo scheme
called Gibbs Sampling. wRACOG will be a modification of RACOG that aims to improve a single classifier provided as parameter.

The next algorithm we will code is RWO. RWO finds its foundation on a statistics theorem that we will prove, that guarantees
us that under asymptotic assumptions and generation scheme, our newly generated instances will preserve the original mean
and variance of the training data set.

PDFOS will be another of the algorithms selected. It is based on multivariate Gaussian kernel density estimations. We will
provide an introduction to the kernel density estimators pointing out their relationship with histograms and we will give
a measure of the error of the estimation, the Mean Integrated Squared Error. PDFOS will try to adapt the bandwith parameter
for a sum of multivariate Gaussian kernels, that is, a multiplicative constant for the unbiased covariance of the minority
training samples. A gradient descendent method will be used to optimize such parameter.

The last algorithm we are going to study is NEATER, a filtering algorithm to clean up noisy instances created as a result of
oversampling. NEATER is based on game theory and Nash profile equilibriums, that guarantee that if an instance could play
two possible roles, it will tend to stabilize as one of them, giving a series of payoffs for itself and its neighbour 
instances.

\section*{Developed software}
An implementation for those algorithms will be made using the \R programming language combined with \texttt{C++} chunks, and 
using \texttt{Armadillo} library that provides vectorized operations under \texttt{C++}. \R language provides a trade-off 
between ease of use for the users and speed of the code, the second one will be used to speed up our algorithms when 
possible. As a result, we will develop a package called \texttt{imbalance} to fill the lack of implementation of the 
described algorithms in \R.

It is important to note not only the code, but the methodology used to develop the code: under a GPLv2 or later license,
allocated in a public Github repo and with continuous integration to provide unit testing and periodic updates of
online web documentation page.

\section*{Experiments on small disjuncts}
Finally, we will conduct a simple experiment on small disjuncts problem, which arise as a direct result of class imbalance
problem. The experiment will somehow show that when we overcome the imbalance of the datasets, small disjuncts tend to
vanish.

We propose a simple measure of the small disjuncts which a dataset has: given a C4.5 tree that labels a set of examples, we 
will consider a small disjunct all those leaves which have less than a given threshold examples. We will also measure the
mean size of the leaves in terms of examples classified, sometimes called coverage.

\paragraph{Keywords}
Machine learning, PAC learning, imbalanced classification, \textit{oversampling}, \textit{small disjuncts}

\setcounter{tocdepth}{1}
\tableofcontents

\break 

% *************************************************************************************************************
% Mainmatter
% *************************************************************************************************************
\pagenumbering{arabic}
\setcounter{page}{1}


\makeatletter
  \def\input@path{{./chapters/}}
\makeatother


\chapter*{Introducción}
  El material principal en el que se ha basado la sección de matemáticas ha sido \cite{understanding_ml}, 
  que proporciona una detallada formalización de los fundamentos matemáticos del Aprendizaje Automático.
\chapter*{Objetivos}

% *************************************************************************************************************
% Parte de matemáticas
% *************************************************************************************************************
\part{Matemáticas}  
  \chapter{Teoría de la probabilidad}
    \section{$\sigma$ álgebras, anillos y semianillos}

Sea en lo que sigue $X$ un conjunto. Dados $A_i \subseteq X$, para $i=1, \ldots n$, notamos $\sum_{i=1}^n A_i$
a su unión disjunta, esto es $\bigcup_{i=1}^n A_i$ con $A_i \cap A_j = \emptyset$ cuando $i\neq j$.

\begin{definition} \textbf{Conjunto potencia}
 Definimos el conjunto potencia de $X$ a $\mathcal{P}(X):= \{A: A\subseteq X\}$.
 
 Como notación alternativa para $\mathcal{P}(X)$ usaremos $2^X$.
\end{definition}


\begin{definition} \textbf{$\sigma$-álgebra}

 $\Sigma \subseteq \mathcal{P}(X) = \{A: A\subseteq X\}$ es $\sigma$-álgebra de conjuntos sobre $X$ si se verifica:
 
 \begin{enumerate}[i]
  \item $X \in \Sigma$
  \item $\Sigma$ es cerrado para complementarios: Sea $A\in \Sigma$, entonces $A^c = X\setminus A \in \Sigma$
  \item $\Sigma$ es cerrado para uniones numerables: Sean $\{A_n\}_{n\in\mathbb{N}} \subseteq \Sigma$, entonces: 
  \[\underset{n\in\mathbb{N}}{\bigcup} A_n \in \Sigma\]
 \end{enumerate}
\end{definition}

\begin{fact}
 Sea $\Sigma$ $\sigma$-álgebra sobre $X$. Entonces:
 
 \begin{enumerate}[i]
  \item $\emptyset \in \Sigma$
  \item $\Sigma$ es cerrada para intersecciones: dados $A,B \in \Sigma$, entonces $A\cap B \in \Sigma$
  \item $\Sigma$ es cerrada para diferencias: dados $A,B \in \Sigma$, entonces $A\setminus B \in \Sigma$
 \end{enumerate}
 
 \label{fact:propsigma}
\end{fact}

\begin{proof}
 Se deducen fácilmente escribiendo $\emptyset = X^c$, $A\cap B = ((A\cap B)^c)^c = (A^c \cup B^c)^c$ y $A\setminus B = A\cap B^c$
 y usando las hipótesis de $\sigma$-álgebra.
\end{proof}



\begin{definition} \textbf{Semianillo en $X$}

 $S\subseteq \mathcal{P}(X)$ es subanillo si verifica:
 
 \begin{enumerate}[i]
  \item $\emptyset \in S$
  \item $A,B \in S$ entonces $A\cup B \in S$
  \item $A,B \in S$ entonces existen $A_1 \ldots A_n \in S$ verificándose $A\setminus B = \sum_{i=1}^n A_i$
 \end{enumerate}
\end{definition}

\begin{example}
 \begin{definition}
  Sean $X_1, X_2$ conjuntos, $\Sigma_i$ $\sigma$-álgebra sobre $A_i$. 
  
  Dados $A_1 \in \Sigma_1, A_2 \in \Sigma_2$ arbitrarios, definimos el rectángulo de lados $A_1$ y $A_2$ como:
  
  \[Rec(A_1, A_2) = A_1 \times A_2\]
 \end{definition}
 
 
 La clase de rectángulos $Rec = \{Rec(A_1, A_2): A_i \in \Sigma_i\}$ es un semianillo en $X_1 \times X_2$,
 ya que dados $R_1 = A_1 \times A_2 \in Rec, R_2 = B_1 \times B_2 \in Rec$ arbitrarios:
 
 \begin{enumerate}
  \item $\emptyset \in \Sigma_i$, y $\emptyset \times \emptyset = \emptyset$
  \item $R_1 \cap R_2 = (A_1 \cap B_1) \times (A_2 \cap B_2)$, donde $A_i \cap B_i \in \Sigma_i$ por ser 
  $\Sigma_i$ cerrada bajo intersecciones.
  \item $R_1 = A_1 \times A_2 \in Rec, R_2 = B_1 \times B_2 \in Rec$, entonces 
  $R_1 \setminus R_2 = \{(x,y): (x,y) \in A_1 \times A_2, (x,y) \notin B_1 \times B_2)\}$
  
  Es decir $R_1 \setminus R_2 = Rec(A_1\setminus B_1, A_2) \cup Rec(B_1, A_2\setminus B_2)$ y 
  $A_i \setminus B_i \in \Sigma_i$ por ser $\Sigma_i$ cerrada bajo diferencias.
 \end{enumerate}

\end{example}


\begin{definition} \textbf{Anillo en $X$}

 $R\subseteq \mathcal{P}(X)$ es anillo si verifica:
 
 \begin{enumerate}[i]
  \item $\emptyset \in R$
  \item $A,B \in R$ entonces $A\cap B \in R$
  \item $A,B \in R$ entonces $A\setminus B \in R$
 \end{enumerate}
\end{definition}


\begin{fact}
 Toda $\sigma$-álgebra es anillo. Todo anillo es semianillo.
\end{fact}

\begin{proof}
 Que $\sigma$-álgebra es más fuerte que anillo quedó probado en la proposición $\ref{fact:propsigma}$
 
 Sea ahora $R \subseteq X$ anillo y veamos que es semianillo.

 Como $A\cap B = A\setminus (A\setminus B)$, se deduce la segunda condición de la definición de semianillo.
 
 Sean $A, B \in R$, tomando $A_1 = A\setminus B \in R$, entonces se verifica la tercera condición de semianillo.
\end{proof}

\begin{counterex}
 Veamos un contraejemplo de que no todo semianillo es anillo:
 
 $\Sigma = \{\emptyset, [0,1], [1,2], [0,2]\}$ es $\sigma$-álgebra sobre $[0,2]$, y $[0,1]^2$ y $[1,2]^2$ son rectángulos
 en $[0,2]^2$, pero $[0,1]^2 \cup [1,2]^2$ no puede ser escrito como producto de $A,B \in \Sigma$.
 
 Y uno de que no todo anillo es $\sigma$-álgebra:
 
 Sea $S = \{A\subseteq \mathbb{N}: |A| < \infty\}$. Entonces se puede comprobar fácilmente que es anillo (y semianillo)
 de $\mathbb{N}$, pero no es cerrado para complementarios porque $\mathbb{N}$ no es finito.
\end{counterex}


\begin{definition}
 Sea $\mathcal{A} \subseteq \mathcal{P}(X)$. Llamamos anillo generado por $\mathcal{A}$ y lo notamos $R(A)$ al
 menor anillo que contiene a $A$:
 
 \[R(A) = \bigcap_{\begin{array}{c}R \textrm{ anillo en } X\\ A\subseteq R \end{array}} R\]
\end{definition}

Es trivial probar que la definición es correcta ($R(A)$ es anillo), ya que dados dos conjuntos $A,B \in R(A)$, 
entonces $A, B \in R$ para todo $R$ anillo contenido a $A$.

\begin{fact}
 Sea $S$ un semianillo en $X$. Entonces:
 
 \[R(S) = \{A: A=\sum_{i=1}^n A_i, n\ge 1, A_i \in S\}\]
 
 \label{claim:semiring}
\end{fact}

\begin{proof}
 LLamamos $\{A: A=\sum_{i=1}^n A_i, n\ge 1, A_i \in S\} = R$. Es claro que $S\subseteq R$.
 
 Sean $A = \sum_{i=1}^n A_i, B = \sum_{i=1}^k B_i \in R$ no nulos. Entonces $A\cap B = \sum_{i,j} (A_i \cap B_j)$ y $R$
 es cerrado para intersección de elementos. Además:
 
 \[A\setminus B = \cup_{i=1}^n \cap_{j=1}^p B_j^c = \cap_{j=1}^p \left(\cup_{i=1}^n (A_i \cup B_j)\right) = \cap_{j=1}^p \left(\sum_{i=1}^n (A_i \cup B_j)\right)\]
 
 Luego aplicando que $R$ es cerrado para intersecciones, llegamos también a que lo es para diferencias.
 
 Por último, como $A\cup B = A \setminus B \sum B$, $A \cup B \in R$, y como $R(S)$ debe contener por definición
 las uniones de elementos de $S$, en particular $R\subseteq R(S)$, pero hemos probado que $R$ es anillo, y $R(S)$
 es el menor anillo que contiene a $S$, luego $R(S) = R$.
\end{proof}


\section{Medidas y extensión de medidas}

\begin{definition} \textbf{Medida}

 Sea $\mathcal{A} \subseteq \mathcal{P}(X)$. Llamamos medida sobre $\mathcal{A}$ a cualquier función 
 $\mu: \mathcal{A} \rightarrow [0, +\infty]$ verificando:

 \begin{enumerate}[i]
  \item $\mu(\emptyset) = 0$
  \item Dados $A_n \in \mathcal{A}$ tales que $A = \sum_{i=n}^{+\infty} A_n \in \mathcal{A}$, entonces 
  $\mu(A)= \sum_{i=n}^{+\infty} \mu(A_n)$
 \end{enumerate}
\end{definition}


\begin{theorem}
 Sea $S$ semianillo en $S$, $\mu: S \rightarrow [0,+\infty]$ medida en $S$. Entonces existe una única medida 
 en $R(S)$, $\bar{\mu}: R(S) \rightarrow [0,+\infty]$ verificando $\bar{\mu}_{|S} = \mu$.
\end{theorem}

\begin{proof}
 Recordamos que $R(S) = \{A: A=\sum_{i=1}^n A_i, n\ge 1, A_i \in S\}$ por la proposición $\ref{claim:semiring}$,
 
 Es claro que $\bar{\mu}$ debería cumplir: $\bar{\mu}(\sum_{i=1}^n A_i) = \sum_{i=1}^n \mu(A_i)$ con $A_i \in S$.
 
 Veamos que $\bar{\mu}$ no depende de cómo escribamos un conjunto, esto es, dados $A = \sum_{i=1}^n A_i = \sum_{j=1}^k B_j$,
 veamos que $\bar{\mu}$ está bien definida para $A$.
 
 Como $A_i = A_i \cap A = A_i \cap \left(\cup_{j=1}^k B_j\right) = \cup_{j=1}^k A_i \cap B_j$, donde $A_i \in S$ y
 $A_i \cap B_j \in S$ por ser $S$ semianillo. Además como los $A_i$ son disjuntos, los $A_i\cap B_j$ también.
 Aplicando que $\mu$ es medida sobre $S$:
 
 \[\mu(A_i) = \sum_{j=1}^k \mu(A_i \cap B_j) \implies \sum_{i=1}^n \mu(A_i) = \sum_{i}^n \sum_{j=1}^k \mu(A_i \cap B_j)\]
 
 Análogamente podemos probar:
 
 \[\sum_{j=1}^k \mu(B_j) = \sum_{i}^n \sum_{j=1}^k \mu(A_i \cap B_j)\]
 
 Luego $\bar{\mu}$ está bien definida. Además $\bar{\mu} (\emptyset) = 0$ por ser $\mu(\emptyset) = 0$.
 
 Falta probar la $\sigma$-aditividad de $\bar{\mu}$ y por construcción, habríamos llegado al resultado buscado.
 
 Sean $A_n = \sum_{i=1}^{k_n} A_i^n \in R(S)$ con $A_i^n \in S$ y $A = \sum_{i=1}^{+\infty} A_n \in R(S)$. 
 Por ser $A \in R(S)$, tendríamos que podemos reescribir $A = \sum_{j=1}^k B_j$ con $B_j \in S$.
 
 Fijamos un $B_p$ y tenemos:
 
 \[B_p = B_p \cap A = \cup_{n=1}^{+\infty} \left(\cup_{i=1}^{n_k} A_i^k \cap B_p\right)\]
 
 Es decir, hemos reescrito $B_p$ como una unión numerable de $A_i^k \cap B_p$, disjuntos por serlo los $A_i$, 
 luego $B_p = \cup_{m\ge 1} C_m$, con $C_m \in S$.
 
 Aplicando $\sigma$-aditividad en $S$, llegamos a $\mu(B_p) = \sum_{n=1}^{+\infty} \sum_{i=1}^{n_k} \mu(A_i^k \cap B_p)$
 
 Además, como podemos reescribir $A_i^{n_k} = \sum_{j=1}^k A_i^{n_k} \cap B_j$, tendríamos:
 
 \begin{align*}
 \mu(A) &= \sum_{j=1}^k \sum_{n=1}^{+\infty} \sum_{i=1}^{n_k} \mu(A_i^k \cap B_j) = 
           \sum_{n=1}^{+\infty} \sum_{i=1}^{n_k} \sum_{j=1}^k \mu(A_i^k \cap B_j) = \\
        &= \sum_{n=1}^{+\infty} \sum_{i=1}^{n_k} \mu(A_i^k) =
           \sum_{n=1}^{+\infty} \mu \left(\sum_{i=1}^{n_k} A_i \right)
 \end{align*}
\end{proof}

\begin{definition} \textbf{Medida exterior}
 Sea $\mu^\ast : \mathcal{P}(X) \rightarrow [0, +\infty]$ verificando:
 
 \begin{enumerate}[i]
  \item $\mu^\ast(\emptyset) = 0$
  \item Monotonía: $A\subseteq B$ entonces $\mu^\ast(A) \le \mu^\ast(B)$
  \item $\sigma$-subaditividad $\mu^\ast \left(\bigcup_{n=1}^{+\infty} A_n \right) \le \sum_{n=1}^{+\infty} \mu^\ast (A_n)$
 \end{enumerate}
 
 Entonces $\mu^\ast$ se dice medida exterior sobre $X$.
\end{definition}

\begin{definition}
 Sea $\mu^\ast$ una medida exterior sobre $X$. Definimos la $\sigma$-álgebra asociada a $\mu^\ast$ como:
 
 \[\Sigma(\mu^\ast) = \{A\subseteq X: \mu^\ast(T) = \mu^\ast(T\cap A) + \mu^\ast(T\cap A^c), \forall T\subseteq X\}\]
\end{definition}

Queda comprobar que la definición de $\Sigma(\mu^\ast)$ es correcta y es de hecho una $\sigma$-álgebra.

\begin{theorem}
 Sea $\mu^\ast : \mathcal{P}(X) \rightarrow [0, +\infty]$ una medida externa sobre $X$. Entonces $\Sigma(\mu^\ast)$
 es una $\sigma$-álgebra sobre $X$ y $\mu^\ast_{|\Sigma(\mu^\ast)}$ es una medida sobre $\Sigma(\mu^\ast)$
 
 \label{th:outer-to-measure}
\end{theorem}

\begin{proof}
 Empezamos viendo que $\Sigma = \Sigma(\mu^\ast)$ es una $\sigma$-álgebra.
 
 \[\mu^\ast(T) = \mu^\ast(T\cap X) + \mu^\ast(T\cap \emptyset) = \mu^\ast(T), \quad \forall T\subseteq X\]
 
 Luego $X \in \Sigma$. También es trivial comprobar que dado $A\in \Sigma$, entonces $A^c \in \Sigma$.
 
 Veamos que $\Sigma$ es cerrada para intersecciones: sean $A, B\in \Sigma$. Usaremos:
 
 \begin{align*}
  T\cap A^c = T\cap (A\cap B)^c \cap A^c\\
  T\cap A \cap B^c = T \cap (A\cap B)^c \cap A
 \end{align*}

 Como $A\in \Sigma$, tenemos $\mu^\ast(T) = \mu^\ast(T\cap A) + \mu^\ast(T\cap A^c)$
 
 Por otro lado, como $B\in \Sigma$, tenemos $\mu^\ast(T\cap A) = \mu^\ast(T\cap A\cap B) + \mu^\ast(T\cap A \cap B^c)$
 
 Es decir:
 
 \begin{align*}
  \mu^\ast(T) &= \mu^\ast(T\cap A^c) + \mu^\ast(T\cap A \cap B) + \mu^\ast(T\cap A \cap B^c) = \\
  &= \mu^\ast(T\cap (A\cap B)^c \cap A^c) + \mu^\ast(T \cap (A\cap B)^c \cap A) +\\
  &+ \mu^\ast(T\cap A \cap B) = \mu^\ast(T\cap (A \cap B)^c) + \mu^\ast(T\cap A \cap B)
 \end{align*}

 Donde en la última igualdad se ha usado que $A\in \Sigma$.
 
 Como podemos escribir $A\setminus B = A\cap B^c$, y $A\cup B = (A^c \cap B^c)^c$ con $A, B \in \Sigma$ que
 hemos probado que es cerrado para intersecciones y complementarios, $\Sigma$ es también cerrado para
 uniones y diferencias.
 
 Dados $A_n \in \Sigma$, como podemos escribir $\cup_{n\ge 1} A_n$ como $\cup_{n\ge 1} B_n$ con los $B_n$ 
 disjuntos, definidos como $B_1 = A_1$, y $B_{n+1} = A_{n+1} \left(\setminus{\cup_{i=1}^n A_i}\right)$, 
 podemos limitarnos a estudiar uniones disjuntas.
 
 De hecho, dados $B,C \in \Sigma$ disjuntos:
 
 \begin{align*}
  \mu^\ast(T \cap (B\cup C)) &= \mu^\ast(T \cap(B\cup C) \cap B) + \mu^\ast(T \cap(B\cup C) \cap B^c) =\\
                             &= \mu^\ast(T \cap B) + \mu^\ast(T \cap C)
 \end{align*}

 Sean $\{B_n\}$ conjuntos disjuntos de $\Sigma$, y llamamos $J = \sum_{n=1}^{+\infty} B_n$, $J_k = \sum_{n=1}^{k} B_n$
 
 Hemos visto que $\Sigma$ es cerrada para uniones finitas, luego $J_k \in \Sigma$ para todo $k\in \mathbb{N}$.
 Fijamos $T\in \mathcal{P}(X)$
 
 \begin{equation}
  \mu^\ast(T) = \mu^\ast(T\cap B + T\cap B^c) \underset{\textrm{monotonía + subaditividad}} \le 
  \mu^\ast(T\cap B) + \mu^\ast(T\cap B^c) \le \mu^\ast(T\cap B) + \sum_{n=1}^{+\infty}\mu^\ast(T\cap B_n))
  \label{eqn:ineqT} \tag{*}
 \end{equation}
 
 Si $\mu^\ast(T) = +\infty$, entonces $\mu^\ast(T) \le \mu^\ast(T) = \mu^\ast(T\cap B^c) = +\infty$, y se da
 la igualdad.
 
 Si $\mu^\ast(T) < +\infty$, tenemos $\mu^\ast(T) \ge \mu^\ast(T\cap J_k) = \sum_{n=1}^k \mu^\ast(T\cap B_n)$
 para $k$ arbitrario, y por tanto $\mu^\ast(T) \ge \sum_{n=1}^{+\infty} \mu^\ast(T\cap B_n)$. Fijado $\epsilon > 0$
 existe $N\in \mathbb{N}$ verificando:
 
 \[\sum_{n=1}^{+\infty} \mu^\ast(T\cap B_n) \le \sum_{n=1}^N \mu^\ast(T\cap B_n) + \epsilon\]
 
 Es decir, tendríamos, desde \eqref{eqn:ineqT}:
 
 \begin{align*}
   \mu^\ast(T) \le \mu^\ast(T\cap B) + \sum_{n=1}^{+\infty}\mu^\ast(T\cap B_n) \le 
   \mu^\ast(T\cap J_N) + \sum_{n=1}^{N}\mu^\ast(T\cap B_n) + \epsilon \le  \mu^\ast(T) + \epsilon
 \end{align*}

 Pero $\epsilon \ge 0$ era arbitrario, luego:
 
 \[\mu^\ast(T) \le \mu^\ast(T\cap B) + \sum_{n=1}^{+\infty}\mu^\ast(T\cap B_n) \le \mu^\ast(T)\]
 
 Es decir: $\mu^\ast(T) = \mu^\ast(T\cap B^c) + \mu^\ast(T\cap B^c)$ y $B\in \Sigma$, y $\Sigma$ es $\sigma$-álgebra
 sobre $X$.
 
 Tomando $T=B$, llegamos a $\mu^\ast(B) = \sum_{n=1}^{+\infty}\mu^\ast(T\cap B_n)$, luego $\mu|_{\Sigma}$ es
 medida sobre $\Sigma$.
 
\end{proof}


\begin{theorem} \textbf{Teorema de extensión de Caratheodory}

 Sea $S \subseteq \mathcal{P}(X)$ un semianillo, $\mu:S \rightarrow [0,+\infty]$ medida en $S$. Entonces existe
 una única medida $\bar{\mu}:\sigma(S) \rightarrow [0,+\infty]$ verificando $\bar{\mu}_{|S} = \mu$
\end{theorem}

\begin{proof}
 Si existiese $\bar{\mu}$ debería verificar, por monotonía, que $\bar{\mu}(T) \le \sum_{n=1}^{+\infty} \bar{\mu}(A_n)$, 
 donde $T \subseteq \cup_{n\ge 1} A_n$ y $A_n\in R$ son disjuntos.
 
 Definimos, para cualquier $T\in \mathcal{P}(X)$:
 
 \[\mu^\ast(T) = \inf\left\{\sum_{n=1}^{+\infty} \mu(A_n), \cup_{n\ge 1} A_n \supseteq T,
                 A_n\in S \forall n\in\mathbb{N}\right\}\]
                 
 Vamos a ver que $\mu^\ast$ es medida exterior sobre $X$.
 
 Claramente, como $\mu \ge 0$, $\mu^\ast \ge 0$, y $\mu^\ast(\emptyset) = \mu(\emptyset) = 0$
 
 Dado $B\subseteq A \subseteq X$, tenemos que si $\cup_{n\ge 1} A_n \supseteq A$ con $A_n\in S \forall n\in\mathbb{N}$,
 entonces $\cup_{n\ge 1} A_n \supseteq B$ y de la definición de $\mu^\ast$ deducimos $\mu^\ast(A) \le \mu^\ast(B)$.
 Hemos probado la monotonía de $\mu^\ast$.
 
 Sean ahora $\{A_n\}$ subconjuntos de $X$, donde para cada $n\in \mathbb{N}$.
 
 Si existe $A_n$ tal que $\mu^\ast(A_n) = +\infty$, entonces por monotonía 
 $\mu^\ast(\cup_{n\ge} A_n) \ge \mu^\ast(A_n) = +\infty$, y por otro lado 
 $\sum_{n=1}^{+\infty} \mu^\ast(A_n) \ge \mu^\ast(A_n) = +\infty$, es decir:

 \[\mu^\ast(\cup_{n\ge} A_n) = \sum_{n=1}^{+\infty} \mu^\ast(A_n) = +\infty\]
 
 Caso de que para todo $n$ se tenga $\mu^\ast(A_n) < +\infty$, fijamos $\epsilon > 0$
 
 Dado$n\in \mathbb{N}$, por ser $\mu^\ast(A_n)$ un ínfimo, deben existir $\{A_p^n\}_{p\ge 1}$ con 
 $A_p^n \in S$ para cualquier $p\in\mathbb{N}$ verificando $\cup_{p\ge 1} \supseteq A_n$ y que
 $\sum_{p=1}^{+\infty} \le \mu^\ast(A_n) + \frac{\epsilon}{2^n}$
 
 Es decir, tenemos $\cup_{n\ge 1} \subseteq \cup_{n\ge 1} \cup_{p\ge 1} A_p^n$ unión numerable
 de conjuntos $A_p^n$ de $S$, luego:
 
 \[\mu^\ast(\cup_{n\ge 1} A_n) \le \sum_{n=1}^{+\infty} \sum_{p=1}^{+\infty} \le 
   \sum_{n=1}^{+\infty}\mu^\ast(A_n) + \sum_{n=1}^{+\infty}\frac{\epsilon}{2^n} = 
   \sum_{n=1}^{+\infty}\mu^\ast(A_n) + \epsilon\]
 
 Pero $\epsilon > 0$ era arbitrario, es decir: 
 \[\mu^\ast(\cup_{n\ge 1} A_n) \le \sum_{n=1}^{+\infty}\mu^\ast(A_n)\]
 
 Es decir $\mu^\ast$ es medida exterior en $X$, y podemos tomar $\bar{\mu}$ medida sobre $\Sigma(\mu^\ast)$ por
 el teorema \ref{th:outer-to-measure}.
 
 Falta comprobar que $\bar{\mu}_{|\Sigma(S)}$ es medida sobre $\Sigma(S)$ y que es la única extensión posible de $\mu$.
\end{proof}



\section{Espacio medible, de probabilidad}

\begin{definition} \textbf{Espacio medible}

 Sea $X$ un conjunto, $\Sigma$ una $\sigma$-álgebra de conjuntos sobre $X$. A la tupla $(X,\Sigma)$ la llamamos
 espacio medible. A los elementos de $\Sigma$ los llamamos conjuntos medibles.
\end{definition}


\begin{definition} \textbf{Espacio de medida}

 Sea $(X, \Sigma)$ espacio medible, y $\mu: \Sigma \rightarrow [0,\infty]$ medida. A la tupla $(X, \Sigma, P)$ 
 la llamamos espacio de medida.
\end{definition}


\begin{definition} \textbf{Espacio de probabilidad}
 Sea $(X, \Sigma, P)$ espacio de medida. Entonces lo llamamos espacio de probabilidad si $P(\Sigma)\subseteq [0,1]$
\end{definition}


\begin{definition} \textbf{Distribución de probabilidad}

 Sea $(X, \Sigma, P)$ espacio de probabilidad. Llamamos a la tupla $\dist = (\Sigma,P)$ distribución sobre $X$. 
 Si $\dist$ es distribución sobre $X$, lo notamos $x\sim \mathcal{D}$
\end{definition}


\section{Desigualdades de concentración}
\begin{lemma} \textbf{Desigualdad de Hoeffding}

 Sean $(X_1, \ldots X_m)$ una muestra aleatoria simple de una variable $X$, 
 $\bar{X} = \frac{1}{m} \sum_{i=1}^m X_i$ con $E[\bar{X}] = \mu$ y $P[a \le X_i \le b] = 1, i=1, \ldots m$. 
 Entonces para todo $\epsilon > 0$

 \[P\left[\left| \bar{X} - \mu \right| > \epsilon \right] \le 2e^{-2m \left(\frac{\epsilon}{b-a}\right)^2}\]
\end{lemma}
  \chapter{Modelo matemático del aprendizaje automático}
    \section{Motivación}

El objetivo del aprendizaje automático es convertir datos en conocimiento a través de un razonamiento inductivo, de manera que
proporcionándole datos a una máquina seamos capaces de extraer un conocimiento (una generalización de los datos que nos permita
inferir información a partir de nuevos datos). Surge la pregunta de por qué es necesario el aprendizaje automático o 
\textit{machine learning}, si la estadística también se encarga de obtener conocimiento a partir de unos datos.

\subsection{¿Por qué necesitamos \textit{machine learning}?}
\begin{enumerate}[i]
 \item Para resolver \textbf{tareas que requieren automatización}: entran dentro de esta categoría tanto aquellas tareas para
 las que no existe una axiomatización o un conocimiento exacto, como pueden ser el reconocimiento de dígitos o de voz, como 
 aquellas tareas que requieren del análisis de un gran número de datos, y quedan fuera de la capacidad humana para realizar
 un análisis estadístico manual. En el primer caso necesitamos apoyarnos en conocimiento auxiliar (por ejemplo, 
 un conjunto de dígitos o de muestras de voz preetiquetadas con los que poder comparar muestras sin etiquetar/clasificar); 
 en el segundo, se hace necsario el uso de una máquina para poder extraer conocimiento de todos los datos.
 
 \item \textbf{Tareas que requieren adaptatividad}: si cambian los datos de entrada, necesitamos que los algoritmos se readapten
 a ellos, y no tengamos un conocimiento rígido, sino que pueda cambiar/mejorar en función de la entrada.
\end{enumerate}

\subsection{Áreas relacionadas con el aprendizaje}
Entre las áreas relacionadas con el aprendizaje automático, cabe mencionar:

\begin{enumerate}[i]
 \item \textbf{Inteligencia Artificial}
 \item \textbf{Algorítmica}: debemos analizar el tiempo asintótico de los algoritmos mediante los que aprende la máquina.
 \item \textbf{Inferencia}: entre las diferencias que podemos mencionar con la estadística convencional, destaca la necesidad de 
 programar las tareas, dado el volumen de datos con el que normalmente se trabaja, mientras que en muchos análisis estadísticos basta 
 lápiz y papel. También destaca la \textbf{independencia respecto a distribución} con la que se trabaja (no se asume una distribución
 determinada sobre los datos). La principal diferencia del aprendizaje automático respecto a la inferencia es que la inferencia
 se encarga de comprobar la validez de las hipótesis que propone el estadista, mientras que el algoritmo de 
 \textit{machine learning} genera hipótesis para unos datos determinados, con unas ciertas condiciones de aproximación y error.
 \item \textbf{Álgebra lineal}
 \item \textbf{Optimización de algoritmos}
\end{enumerate}

\subsection{Ejemplo práctico}\label{sec:first-ex}
Pensemos en un ejemplo: tenemos clientes de un banco que quieren solicitar un préstamo, y a todos se les categoriza
el tamaño del préstamo que quieren solicitar(cómo de grande es su importe) y su nivel de ingresos. Estas variables se miden 
en una escala de $0$ a $1$ donde $1$ es el nivel máximo. Queremos etiquetar a cada cliente como $1$:conceder préstamo o 
$0$:no conceder préstamo.

Dado un histórico de $m\in \mathbb{N}$ clientes a los que se les concedieron y devolvieron o no préstamos, tenemos una tupla 
$((x_1, y_1), \ldots (x_m, y_m))$ donde $x_i = ((x_i)_1, (x_i)_2) \in [0,1]^2, y_i \in \{0,1\}$. Asimismo $(x_i)_1$ representa el 
tamaño del préstamo solicitado y $(x_i)_2$ representa el nivel ingresos mensuales. Tenemos a los clientes clasificados en 
función de si devolvieron los préstamos o no.

Queremos encontrar una función que ofrezca una predicción sobre cualquier cliente, para minimizar posibles pérdidas del banco, es
decir, buscamos una $f:[0,1]^2 \rightarrow \{0,1\}$ que llamaremos predicción.

Asumimos que los datos de los clientes de que disponemos no van a tener una distribución determinada y
van a ser idéntica e independientemente distribuidos (i.i.d.). El histórico siempre va a crecer, y no sabemos cómo va a hacerlo, queriendo aprovechar
al máximo la información del mismo.

También asumiremos que tenemos una clase de predicciones $H \subseteq \{0,1\}^{[0,1]^2}$, de entre las que existe
una óptima. Podríamos limitar la clase de predicciones sobre la que buscamos como subrectángulos de $[0,1]^2$ por ejemplo, 
esto es, funciones:

\[h_{a,b,c,d} = \mathds{1}_{[a,b]\times[c,d]}, \qquad [a,b]\times [c,d] \subseteq [0,1]^2\]

\img{./imgs/rect-ex.png}{0.85}

También necesitaremos definir una medida de acierto: ¿cómo de buena es una predicción?.

\section{Definiciones básicas}
\label{sec:defs}

Damos unas notaciones/definiciones básicas que utilizaremos de aquí en adelante, en base a lo descrito en \ref{sec:first-ex}:

\begin{itemize}
\item \textbf{Dominio}: $X$. Llamamos \textbf{instancia} a $x\in X$

\item \textbf{Conjunto de etiquetas}: $Y$ consideramos $\{0,1\}$, lo que nos restringe de momento al paradigma de clasificación binaria. 
En ocasiones también usaremos $Y = \{-1,1\}$ para las etiquetas. Al $1$ se le llama clase positiva, y al $0$ o $-1$ clase negativa.

\item \textbf{Verdadero etiquetado}: \sloppy Asumimos la existencia de una función ${f: X \rightarrow Y}$ 
que proporciona la verdadera etiqueta de todas las instancias.

\item \textbf{Generación de instancias}: \fussy Se tiene $x\sim \mathcal{D} = (\Sigma, \prob)$. La distribución de probabilidad nos da 
información sobre la probabilidad de extraer cada posible instancia desde  $x \in X$. 

\item \textbf{Conjunto de entrenamiento}: $S = ((x_1,y_1) \ldots (x_m,y_m)) \in (X \times Y)^m$ 
Nótese que llamarlo conjunto puede dar lugar a confusión, puesto que se trata de una tupla. Notaremos 
$S_x = (x_1, \ldots x_m)$

De momento asumiremos que las etiquetas del conjunto de entrenamiento se corresponden con el verdadero etiquetado: 
$y_i = f(x_i)$, por lo que no podemos tener una instancia con etiquetas diferentes.

La elección de $S_x$ es idéntica e independientemente distribuida, esto es $x_i \sim \mathcal{D}$ para todo $i=1, \ldots, m$.
Lo notamos $S_x \sim \mathcal{D}^m$, o $S \sim \mathcal{D}^m$, por abuso de notación.

\item \textbf{Hipótesis/clasificador/predicción}: cada posible aplicación perteneciente a 
$\{h, h:X \rightarrow Y\} := 2^{X}$. 

\item \textbf{Error del clasificador}: Definimos el error del clasificador, suponiendo 
$\{x\in X : h(x) \neq f(x)\} := [h\neq f] \in \Sigma$ como:

\[L_{\mathcal{D},f}(h) :=  \prob [h \neq f]\]

\item \textbf{Algoritmo de aprendizaje}: Llamamos algoritmo de aprendizaje a cualquier aplicación que tome conjuntos de 
entrenamiento y devuelva hipótesis sobre el problema:

\[A: \underset{m\in \mathbb{N}}{\bigcup} (X\times Y)^m \rightarrow 2^{X}\]

Asumimos que el algoritmo no tiene acceso a la función de verdadero etiquetado $f: X \rightarrow Y$ ni a
la distribución $\mathcal{D}$.
\end{itemize}

Es decir, nos centraremos en el problema de la clasificación, que se trata de un aprendizaje supervisado (conocemos las etiquetas para
el conjunto de entrenamiento), \emph{batch} (recibimos todo el conjunto de entrenamiento, y no sucesivas porciones del mismo en
contraposición al aprendizaje por lotes) y pasivo (el algoritmo no tiene interacción con el usuario).

\subsection{Relación entre hipótesis y conjuntos}
En clasificación binaria, existe una biyección canónica entre la clase de hipótesis y el conjunto potencia de $X$, donde
a cada hipótesis se le asigna su clase positiva:

\begin{equation}
 \begin{array}{rcl} 
  \{h, h:X \rightarrow Y\} & \longrightarrow & \mathcal{P}(X) \\
  h & \longmapsto & X_h := \{x\in X: h(x) = 1\}
 \end{array}
 \label{biyeccion-canonica} 
\end{equation}

Es biyección claramente, lo que justifica que identifiquemos $\{h, h:X \rightarrow Y\}$ con $2^X$, donde $2^X$ suele ser una notación
empleada para referirse a $\mathcal{P}(X)$. Esta biyección nos permitirá trabajar refiriéndonos a $h:X \rightarrow Y$ 
como hipótesis o conjunto.

\subsection{Minimización del riesgo empírico}

\begin{definition} \textbf{Riesgo empírico}

Definimos el riesgo empírico o error empírico como:

\[L_S(h) = \frac{|i\in {1,\ldots, m}: h(x_i) \neq y_i|}{m}\]

\end{definition}

Es decir, es el error del clasificador $h$ sobre el conjunto de entrenamiento $S$. 

\begin{fact}
Sea $H\subseteq 2^X$. Sea $\mathcal{D}$ distribución sobre $\mathcal{X}$. Sea $f \in mathcal{H}$ fija. Entonces para
cualquier $h\in H$:
\[\mathbb{E}_{S\sim \mathcal{D}} [L_S(h)] = L_{\dist,f}(h)\]

\end{fact}

\begin{proof}
  Llamamos $p= \prob [f\neq h]$

  \begin{align*}
  \expect_{S\sim \dist} [L_S(h)] = \sum_{k=0}^m \frac{k}{m} \binom{m}{k} P^k(1-p)^{m-k} & = \sum_{k=1}^m \frac{k}{m} \binom{m}{k} p^k(1-p)^{m-k} =\\
  = \sum_{k=1}^m \binom{m-1}{k-1} p^k(1-p)^{m-k} = \sum_{k=0}^{m-1} \binom{m-1}{k} p^{k+1}(1-p)^{m-1-k} = \\
  = p\cdot \sum_{k=0}^{m-1} \binom{m-1}{k} p^{k}(1-p)^{m-1-k} = p(1+(1-p))^{m-1} = p
  \end{align*}
\end{proof}


\begin{definition} \textbf{Minimizador del riesgo empírico, ERM}

Decimos que un algoritmo $A: \underset{m\in \mathbb{N}}{\bigcup} (X\times Y)^m \rightarrow 2^{X}$ es un $ERM$ 
(\emph{Empirical Risk Minimizer}) si busca una hipótesis cuyo error empírico sea mínimo:

\[A(S) = \argmin_{h\in 2^X} L_S(h)\]
\end{definition}

Cuando notemos $ERM(S)$ nos refereriremos a una propiedad que se verifica para toda la clase de algoritmos que son ERM.

Trivialmente $L_S(f) = 0$ para cualquier $S \in (X \times Y)^m$ conjunto de entrenamiento, por la hipótesis de que existe una
verdadera función de etiquetado. Esto implica que $A(S) = 0$ para $A$ un algoritmo ERM.

Aunque $A(S)$ minimice el error sobre el conjunto de entrenamiento, esto no significa que el error $L_{\mathcal{D},f} (A(S))$ 
sea mínimo. Pensemos en el siguiente ejemplo:

\begin{example}
Sea $X = \mathbb{R}$, $\mathcal{D}$ la distribución uniforme sobre $[0,2]\subset \mathbb{R}$, y la siguiente función:

\[f(x) = \left\{\begin{array}{lcl}
1 && x\in [0,1]\\
0 && x\in \mathbb{R}\setminus [0,1]
\end{array}\right.\]


Sea $S = ((x_1,y_1), \ldots (x_m, y_m))$ un conjunto de entrenamiento de tamaño $m$ sin elementos repetidos y el clasificador:

\[h_S(x) = \left\{\begin{array}{lcl}
y_i && \exists i\in \{1\ldots m\} : x=x_i\\
0 && \nexists i\in \{1\ldots m\} : x=x_i
\end{array}\right.\]

Este clasificador es un ERM, pero $\prob[h_S(x)] = 1/2$, es decir, tiene el mismo nivel de acierto que el 
clasificador idénticamente 1. A este fenómeno lo denominamos \textbf{overfitting}.
\end{example}

\begin{definition} \textbf{Overfitting}

 Decimos que un clasificador $h: X\rightarrow Y$ produce overfitting sobre el conjunto de entrenamiento 
 $S$ si $L_S(h) << L_{\mathcal{D},f}(h)$.
\end{definition}

Necesitamos incorporar por tanto \textbf{conocimiento previo} al problema, de manera que revisitamos la definición de riesgo
empírico.

\subsection{ERM con sesgo inductivo}
Se intenta corregir el riesgo de producir \emph{overfitting} restringiendo el espacio de búsqueda, esto es, la clase de 
hipótesis $H \subseteq 2^X$ desde la que el algoritmo puede escoger un $h: X\rightarrow Y$. Llamamos a esto 
\emph{sesgo inductivo}, puesto que se asume una determinada clase de funciones $H$ en función de las 
características del problema y del conocimiento previo que tenemos del mismo.

\begin{definition} \textbf{ERM con sesgo inductivo}

Decimos que un algoritmo $A: \underset{m\in \mathbb{N}}{\bigcup} (X\times Y)^m \rightarrow 2^{X}$ es un $ERM$ con sesgo 
inductivo hacia $H$ y lo notamos $ERM_H$, si busca una hipótesis en $H$ cuyo error empírico 
sea mínimo, es decir:

\[A(S) = \argmin_{h\in H} L_S(h)\]
\end{definition}

Cuando notemos $ERM_H(S)$ nos refereriremos a una propiedad que se verifica para toda la clase de algoritmos que 
son $ERM_H$.

\begin{definition} \textbf{Hipótesis de factibilidad}

Diremos que $H$ verifica hipótesis de factibilidad respecto a $f\in 2^X$ y $\mathcal{D}$ distribución sobre $X$, si 
$\exists {\bar{h}} \in H$ verificando $L_{D,f}(\bar{h}) = 0$.
\end{definition}

\begin{fact}
Dada $H$ verificando factibilidad respecto a $f\in 2^X$ y $\mathcal{D}$ distribución sobre $X$, entonces siendo 
${\bar{h}} \in H$ tal que $L_{D,f}(\bar{h}) = 0$, se verifica:

\[\mprob \bigg[L_S(\bar{h}) = 0 \bigg] = 1\]

En particular: $\mprob \bigg[L_S(ERM_H(S)) = 0 \bigg] = 1$
\label{fact:ermh}
\end{fact}

  \begin{proof}
  Sea $A$ un $ERM_H$, y además sabemos que por hipótesis de factibilidad existe $\bar{h} \in H$ tal que
  $L_{\mathcal{D},f}(\bar{h}) = \prob[\bar{h}\neq f] = 0$

  Entonces: \[\mprob[L_S(\bar{h}) > 0] = \mprob
  [\exists i: \bar{h}(x_i) \neq f(x_i)] \le \sum_{i=1}^m \prob[h\neq f] = 0\]

  Y como $[L_S(\bar{h}) = 0] \subseteq [L_S(A(S)) = 0]$ por ser $A(S) = \argmin_{h\in \mathcal{H}} L_S(h)$, deducimos:
  \[1 = \mprob[L_S(\bar{h}) = 0] \le \mprob[L_S(A(S)) = 0] \le 1\]
  \end{proof}

El valor $L_{\mathcal{D},f}(ERM_H(S))$ dependerá por tanto del conjunto de entrenamiento $S$, y la elección del
mismo está sometida al azar. 

%Además, necesitamos definir cómo de buena será la predicción.
  \chapter{Aprendizaje PAC}
    \section{PAC cognoscibilidad}
Proporcionamos a continuación una definición que materializa el concepto de que un algoritmo pueda aprender de los datos. Las
siglas PAC derivan de Probablemente Aproximadamente Correcto.

\begin{definition*} \textbf{PAC cognoscible}

Una clase de funciones $H \subseteq 2^X$ es PAC cognoscible sii existen una función 
$m_{H} : ]0,1[^2\rightarrow \mathbb{N}$, llamada complejidad muestral, y un algoritmo $A$ verificando que para todo
$0 < \epsilon, \delta < 1$, para toda distribución $\mathcal{D}$ sobre $X$ y para toda función de 
verdadero etiquetado $f\in H$, dados $m \ge m_H(\epsilon, \delta)$ y $S\sim \mathcal{D}^m$:

\[P_{S\sim \mathcal{D}^m} \bigg[L_{\mathcal{D},f}(A(S)) \le \epsilon \bigg] \ge 1-\delta\]
\end{definition*}

Llamamos a $(1-\delta)$ \textbf{confianza} de la predicción y a $(1-\epsilon)$ \textbf{exactitud}. Estos dos parámetros 
explican el nombre aproximadamente ($\equiv$ confianza) correcto ($\equiv$ exactitud).

La definición de PAC cognoscibilidad nos exige que exista un algoritmo y una función a la que proporcionándole unos requisitos
de confianza y exactitud, nos indique un tamaño del conjunto de entrenamiento tal que el algoritmo pueda aproximar cualquier
distribución y etiquetado, siempre que le ofrezcamos como entrada un conjunto de entrenamiento que satisfaga los 
requisitos de tamaño.

Consideraremos $m_{H}$ única en el sentido de que para cada $(\delta, \epsilon)$ nos devuelva el menor natural
verificando las hipótesis del enunciado.

Nótese que las condiciones exigidas, cumplir la hipótesis de factibilidad y que la hipótesis devuelta deba estar en $H$, 
son muy fuertes. En contraposición, esta definición tiene la virtud de que el tamaño muestral es independiente de la 
distribución. Relajaremos esta definición más adelante con el concepto de PAC agnóstico.

\begin{theorem*} \textbf{Las clases finitas son PAC cognoscibles}

Sea $H \subseteq 2^{X}$ finito. Entonces $H$, bajo hipótesis de factibilidad, es PAC cognoscible con:

\[m_H(\epsilon, \delta) \le \left\lceil \frac{1}{\epsilon}log \left(\frac{|H|}{\delta} \right) \right\rceil\]
\end{theorem*}

  \begin{proof}
  Fijamos $\epsilon \in ]0,1[$. Sea una distribución $\mathcal{D}$ sobre $X$, $m\in \mathbb{N}$ y una función de verdadero 
  etiquetado $f\in H$, tomamos la clase de hipótesis ``malas'':

  \[H_B = \{h\in H: L_{\mathcal{D},f}(h) > \epsilon\}\]

  Sea $A$ un $ERM_{\mathcal{H}}$, entonces:

  \[\underset{S\sim \mathcal{D}^m}{P} [L_{\mathcal{D},f}(A(S)) > \epsilon] \le \underset{S\sim \mathcal{D}^m}{P} 
  [\exists h\in H_B : L_S(h) = 0] \le \sum_{h\in H_B} P [L_S(h) = 0] \]

  La primera desigualdad viene dada por la proposición \ref{fact:ermh}. La segunda, por subaditividad, puesto que:
  
  \[\underset{S\sim \mathcal{D}^m}{P} [\exists h\in H_B : L_S(h) = 0] = \underset{S\sim \mathcal{D}^m}{P} \left(\underset{h\in H_B}{\bigcup} [L_S(h) = 0]\right)\]

  Además, fijada $h\in H_B$, como $L_{\mathcal{D},f}(h) > \epsilon$:

  \begin{align*}
  \underset{S\sim \mathcal{D}^m}{P}[L_S(h) = 0] = \underset{S\sim \mathcal{D}^m}{P}[h(x_i) = f(x_i), i =1,\ldots m\}] =\\
  = \prod_{i=1}^m \underset{x\sim \mathcal{D}}{P}[h = f] = \prod_{i=1}^m (1 - L_{\mathcal{D},f}(h)) \le (1-\epsilon)^m \le e^{-\epsilon m}
  \end{align*}


  Las dos desigualdades probadas, junto a la hipótesis del enunciado, y usando $H_B \subseteq H$ dan lugar a:

  \[P_{S\sim \mathcal{D}^m}[L_{\mathcal{D},f}(h_S) > \epsilon] \le |H|e^{-\epsilon m}\]
  
  Y basta acabar haciendo encontrando $m\in \mathbb{N}$ tal que $|H|e^{-\epsilon m} \le \delta$.
  \end{proof}

¿Hay ejemplos de clases infinitas PAC cognoscibles? La respuesta es afirmativa. Veamos un ejemplo.

\begin{example}
  \begin{definition*} \textbf{Clasificadores de rectángulo}

  La clase de clasificadores de rectángulo en el plano se define por:

  \[H^2_{rec} = \{ h_{a,b,c,d}: a\le b, c\le d\}\]

  donde $h_{a,b,c,d} = \mathds{1}_{[a,b]\times [c,d]}$
  \end{definition*}


  \begin{fact} \textbf{$H_{rec}^2$ es PAC cognoscible}

  Bajo hipótesis de factibilidad, los rectángulos son PAC cognoscibles
  \end{fact}

    \begin{proof}
    Sea $A$ el algoritmo que devuelve el rectángulo más pequeño que engloba a todos los ejemplos positivos del conjunto 
    de entrenamiento $S$. Notamos a dicho rectángulo $R(S)$, $A(S) = \mathds{1}_{R(S)}$.

    Por hipótesis de factibilidad, $\exists R^{\ast} = [a,b]\times [c,d]$ tal que $\bar{h} = \mathds{1}_{R^{\ast}}$ cumple 
    $L_{\mathcal{D},f}(\bar{h}) = 0$, y por proposición \ref{fact:ermh} se tiene 
    $\underset{S\sim \mathcal{D}^m}{P} \bigg[L_S(\bar{h}) = 0 \bigg] = 1$. 
    Dado por tanto $S \in (X\times Y)^m$ tal que $L_S(\bar{h}) = 0$, claramente $\bar{h}(x,1)= 1 = A(S)(x,1)$ para 
    todo $(x,1)\in S$. Por tanto $R(S) \subseteq R^{\ast}$ c.p.d., y se deduce $\bar{h}(x,0)= 0 = A(S)(x,0)$ para 
    todo $(x,0)\in S$. $A$ es por tanto un ERM.

    Fijamos $1 > \epsilon, \delta > 0$.

    Tomamos $R_1 = [a,b^{\ast}] \times [c,d]$ un rectángulo verificando $L_{\mathcal{D},f}(\mathds{1}_{R_1}) = \epsilon/4$, 
    con $a\le b^{\ast} \le b$.

    $R_2= [a^{\ast},b] \times [c,d], R_3=[a,b] \times [c,d^{\ast}], R_4=[a,b] \times [c^{\ast},d]$ se definen 
    de forma análoga, y notamos $\bar{R_i} = R^{\ast} \setminus R_i$

    Fijado $R=R(S)$, supongamos $\forall i : R \cap \bar{R_i} \neq \emptyset$. Entonces:

    \[L_{\mathcal{D},f}(h_R) = \underset{x\sim \mathcal{D}}{P} [h_R \neq f] \le \underset{x\sim \mathcal{D}}{P} 
    \left(\underset{i}{\cup} [h_R \neq f] \cap \bar{R}_i\right) \le \underset{x\sim \mathcal{D}}{P} \left(\underset{i}{\cup} 
    \bar{R}_i\right) \le 4\frac{\epsilon}{4}\]

    La demostración acaba probando que:

    \[\underset{S\sim \mathcal{D}^m}{P} [\exists i : R(S)\cap \bar{R}_i = \emptyset] \le \sum_{i=1}^4 
    \underset{S\sim \mathcal{D}^m}{P}[R(S)\cap \bar{R}_i = \emptyset] = 4(1-\frac{\epsilon}{4})^m \le 4e^{-\epsilon m/4}\]

    y tomando $m > \frac{4}{\epsilon} log \left( \frac{4}{\delta} \right)$ llegamos al resultado buscado.
    \end{proof}
\end{example}


  \chapter{Aprendizaje uniforme}
    \section{Clases de Glivenko-Cantelli}

Empezamos definiendo lo que es un conjunto de entrenamiento $\varepsilon$ representativo.

\begin{definition*}
 Un conjunto de entrenamiento $S$ es $\varepsilon$-representativo de una clase de hipótesis $H$ respecto a una
 distribución $\mathcal{D}$ y una función de pérdida $l$, si $\forall h\in H$ se tiene:
 
 \[|L_S(h) - L_{\dist}(h)| \le \varepsilon\]
\end{definition*}

Intuitivamente, cualquier $ERM_H$ sobre $S$ sería un buen algoritmo de aprendizaje 
si $S$ es $\varepsilon$ representativo para cualquier $\varepsilon \in ]0,1[$ y cualquier distribución. 
Lo formalizamos con la siguiente proposición:

\begin{fact}
 Si $S$ es $\varepsilon$-representativo de $H$ con respecto a una distribución $\mathcal{D}$ y una función de 
 pérdida $l$, entonces se tiene:
 
 \[L_{\dist}(ERM(S)) \le \inf_{h\in H} L_{\dist}(h) + 2\varepsilon\]
 
 \label{fact:epsilon-rep}
\end{fact}

\begin{proof}
 Sea $A$ un $ERM_H$
 
 $L_{\dist}(A(S)) \le L_S(A(S)) + \varepsilon$ por ser $S$ $\varepsilon$-representativa. Pero por definición de 
 $ERM_H$, $\inf_{h\in H} L_S(A(S)) + \varepsilon \le \inf_{h\in H}L_S(h) + \varepsilon$ y aplicando $\varepsilon$-representatividad de nuevo:
 
 \[\inf_{h\in H} L_S(h) + \varepsilon \le \inf_{h\in H} L_{\mathcal{D}}(h) + 2\varepsilon\]
 
 En resumen: $L_{\dist}(A(S)) \le \inf_{h\in H} L_{\mathcal{D}}(h) + 2\varepsilon$, para $A \in ERM_H$ arbitrario.
\end{proof}


\begin{definition*} \textbf{Clase de Glivenko-Cantelli}

Decimos que una clase de hipótesis $H$ es de Glivenko-Cantelli, respecto a un dominio $Z$, y a 
una función de pérdida $l$, si existe una función ${m_{H}^{CU}: ]0,1[^2 \rightarrow \mathbb{N}}$ 
verificando que para todo $0 < \delta, \varepsilon < 1$ y para toda distribución $\dist$ sobre $Z$, siendo 
$S\sim \dist^m$ con $m \ge m_{H}^{CU}(\varepsilon, \delta)$, 
entonces:

\[\mprob [\forall h\in H, |L_S(h) - L_{\dist}(h)| \le \varepsilon] \ge 1-\delta\]
\end{definition*}

Análogamente a lo que ocurría en la definición \ref{def:pac}, podemos considerar $m_H^{CU}$ única, en el sentido de
que es la mínima función que satisface las hipótesis.

\section{Glivenko-Cantelli y APAC cognoscibilidad}

\begin{theorem*}
Sea $H$ una clase de hipótesis de Glivenko-Cantelli, respecto a $\mathcal{D}$ y función de pérdida $l$. 
Entonces es APAC cognoscible con cualquier algoritmo $ERM_H$ y complejidad muestral
$m(\varepsilon, \delta) \le m_{H}^{UC} \left(\frac{\varepsilon}{2}, \delta \right)$ 
\end{theorem*}

  \begin{proof}
  Sea $A$ un $ERM_H$ arbitrario, y $\dist$ una distribución arbitraria sobre $Z$.
  Fijamos $m = m_{H}^{UC} \left(\frac{\varepsilon}{2}, \delta \right)$.

  Sea $S = (z_1, \ldots z_m)$ un conjunto de entrenamiento, verificando que: 

  \begin{equation}
    \forall h\in H, |L_{S}(h)-L_{\dist}(h)| \le \frac{\varepsilon}{2}
    \label{eq:gc-ineq}
  \end{equation}

  Entonces $S$ es $\varepsilon$ representativa para $\dist$ y $l$, y por proposición \ref{fact:epsilon-rep}:

  \begin{equation}
   L_{\dist}(A(S)) \le \inf_{h\in H} L_{\dist}(h) + \varepsilon
   \label{eq:gc-result}
  \end{equation}

  Pero como $\eqref{eq:gc-ineq}$ ocurre con probabilidad (sobre $S$) mayor o igual a $1-\delta$, entonces $\eqref{eq:gc-result}$
  ocurre con probabiliad mayor o igual a $1-\delta$
  
  \end{proof}
  
  
\begin{fact}
Sea $H$ una clase de hipótesis finita, $Z$ un dominio y sea $l : H \times Z \rightarrow [a,b]$ una función de pérdida. Entonces $H$ verifica la propiedad de convegencia uniforme con: 

\[m_{H}^{CU}(\varepsilon, \delta) \le \left\lfloor \frac{log(2|H|/\delta)(b-a)^2}{2\varepsilon^2} \right\rfloor + 1\]
\end{fact}
  \begin{proof}
  Sea $H$ una clase de hipótesis finita.

  Fijamos $0 < \delta, \varepsilon < 1$. Necesitamos encontrar $m\in \mathbb{N}$ verificando:

  \[P_{S\sim \mathcal{D}^m} [\exists h\in H |L_S(h) - L_{\mathcal{D}}(h)| > \varepsilon] < \delta\]

  Partimos de la siguiente desigualdad, que usaremos más adelante, obtenida por subaditividad:

  \[P [\exists h\in H |L_S(h) - L_{\mathcal{D}}(h)| > \varepsilon] \le \sum_{h \in H} P [|L_S(h) - L_{\mathcal{D}}(h)| > \varepsilon]\]
  Fijamos $h \in H$.

  Dado un conjunto de entrenamiento $S_z = (z_1, \ldots z_m)$, recordamos que $L_{\mathcal{D}} (h) = \mathbb{E}_{z\sim \mathcal{D}} [l(h,z)]$ y que $L_{S_z}(h) = \frac{1}{m} \sum_{i=1}^m l(h,z_i)$

  Donde $z_i \sim \mathcal{D}$ y por tanto $\mathbb{E}_{S \sim \mathcal{D}^m} [L_S(h)] = \mathbb{E}_{z \sim \mathcal{D}} [l(h,z)] = L_{\mathcal{D}} (h)$. Además, llamando $X_i = l(h,Z_i)$, por ser $S=(Z_1, \ldots Z_m)$ m.a.s que genera los conjuntos de entrenamiento, se tiene que las $X_i$ son independientes e idénticamente distribuidas, con $P[a \le X_i \le b] = 1$. Estamos en condiciones de aplicar la desigualdad de Hoeffding.

  Por tanto:

  \[P \left[\left| \frac{1}{m} \sum_{i=1}^m X_i - L_{\mathcal{D}} (h) \right| > \varepsilon\right] = P [|L_S(h) - L_{\mathcal{D}}(h)| > \varepsilon] \le 2e^{-2m \left( \frac{\varepsilon}{b-a} \right)^2}\]

  Y por tanto:

  \[P [\exists h\in H |L_S(h) - L_{\mathcal{D}}(h)| > \varepsilon] \le |H| 2e^{-2m \left( \frac{\varepsilon}{b-a} \right)^2}\]

  Despejando $m$ para que $|H| 2e^{-2m \left( \frac{\varepsilon}{b-a} \right)^2} < \delta$ llegamos al resultado buscado.
  \end{proof}

Recordemos hasta ahora el resultado que habíamos obtenido era su carácter PAC cognoscible, donde 
agnósticamente PAC cognoscible y cognoscible con funciones de pérdida 0-1 era un término equivalente. El 
teorema que enunciamos a continuación, deducible a partir del teorema sobre el caracter agnóstico - PAC 
cognoscible de clases de funciones con propiedad de convergencia uniforme, en particular las finitas, 
generaliza el resultado para cualquier funciones de pérdida acotada.

\begin{theorem*}

Sea $H$ una clase de hipótesis finita, $Z$ un dominio y sea $l : H \times Z \rightarrow [a,b]$ una función de pérdida. Entonces $H$ es PAC cognoscible con complejidad muestral:

\[m_{H}( \varepsilon, \delta ) \le \left\lceil \frac{2 log(2|H|/\delta)(b-a)^2}{\varepsilon^2} \right\rceil\]
\end{theorem*}
  \begin{proof}
  Es trivial desde el anterior teorema y el hecho de que convergencia uniforme implica ser agnósticamente PAC cognoscible
  \end{proof}

  \chapter{No Free Lunch}
    \section{Teorema de No Free Lunch}
Veamos que dado un algoritmo de aprendizaje no puede ser el óptimo para aprender todas las distribuciones.

\begin{theorem}
\textbf{Teorema de No Free Lunch}

Sea $A$ cualquier algoritmo de aprendizaje para clasificación binaria con respecto a la función de pérdida 0-1 sobre el dominio $X$. Sea un conjunto de entrenamiento de tamaño $m < |X|/2$. Entonces existe una distribución $\mathcal{D}$ sobre $X \times \{0,1\}$ verificando:

\begin{enumerate}
\item Existe una función $f: X \rightarrow \{0,1\}$ con $L_{\mathcal{D}}(f)=0$
\item $P_{S\sim \mathcal{D}^m} [L_{\mathcal{D}} (A(S)) \ge 1/8] \ge 1/7$
\end{enumerate}
\end{theorem}

\begin{proof}
Sea un conjunto de entrenamiento (consideramos un conjunto y no una secuencia) de tamaño $2m$, $C$. Hay $T = 2^{2m}$ posibilidades de etiquetado del conjunto, esto es, $T$ posibles hipótesis, $f_i: C\rightarrow \{0,1\}$, que vamos a extender a $X$ llamándolas $\bar{f}_i$ de forma que $\bar{f}_{i|C} = f_i$ y $\bar{f}_i(x) = 0 \quad \forall x\in X\setminus C$. Vamos a tomar para cada una de ellas una distribución $\mathcal{D}_i$ definida sobre $X \times \{0,1\}$ definida por:


\[\forall (x,y)\in X \times \{0,1\} \qquad P_{Z\sim \mathcal{D}_i} [Z = (x,y)] = \left\{\begin{array}{ll}
1/|C| & x \in C, y=f_i(x)\\
0     & si \quad no
\end{array}\right.\]

Claramente $L_{\mathcal{D}_i}(f_i) = 0$. Tenemos distribuciones de probabilidad que sólo asignan toda la masa de probabilidad a la marginal en $X$ al conjunto $C$.

Vamos a probar que:

\[\exists i\in \{1, \ldots T\} : \mathbb{E}_{S\sim \mathcal{D}_i^m} [L_{\mathcal{D}_i} (A(S))] \ge \frac{1}{4}\]

Fijamos $i \in \{1, \ldots T\}$. Hay $k = (2m)^m$ posibles tuplas de tamaño $m$, $S_{j}, j=1, \ldots k$ tomadas desde $C$. Siendo $S_j = (x_1, \ldots x_m)$ notamos $S_j^i = ((x_1, f_i(x_1)), \ldots, (x_m, f_i(x_m)))$. Cada $S_j$ tiene la misma probabilidad de ser nuestro conjunto de entrenamiento (extracción de $m$ valores con reemplazamiento desde el conjunto $C$), verificándose:

\[\mathbb{E}_{S\sim \mathcal{D}_i^m} [L_{\mathcal{D}_i} (A(S))] = \frac{1}{k} \sum_{j=1}^k L_{\mathcal{D}_i} (A(S_j^i))\]

Recordando que hemos llamado $k=(2m)^m$, $T=2^{2m}$, se tiene:

\begin{align*}
max_{i \in \{1,\ldots T\}} \frac{1}{k} \sum_{j=1}^{k} L_{\mathcal{D}_i} (A(S_j^i)) &\ge 
       \frac{1}{T} \sum_{i=1}^{T} \frac{1}{k} \sum_{j=1}^{k}  L_{\mathcal{D}_i} (A(S_j^i))   =\\
&=     \frac{1}{k} \sum_{j=1}^{k} \frac{1}{T} \sum_{i=1}^{T}  L_{\mathcal{D}_i} (A(S_j^i)) \ge\\
&\ge min_{j \in \{1, \ldots k\}} \frac{1}{T} \sum_{i=1}^{T}  L_{\mathcal{D}_i} (A(S_j^i))
\end{align*}


Además fijado $j \in \{1,\ldots k\}$:

Sean ${v_r}_{i=r}^p$ los elementos de $C$ no presentes en el conjunto de entrenamiento $S_j$. Claramente, como $|C|=2m$ y $|S_j| = m$ y puede tener elementos repetidos, $p \ge m$

\[L_{\mathcal{D}_i} (A(S^i_j)) = \frac{1}{|C|} \sum_{x\in C} \mathds{1}_{[A(S^i_j)(x) \neq f_i(x)]} = \frac{1}{2m} \sum_{x \in C} \mathds{1}_{[A(S^i_j)(x) \neq f_i(x)]}\]


Por tanto:

\begin{align*}
\frac{1}{T} \sum_{i=1}^{T}  L_{\mathcal{D}_i} (A(S_j^i)) &\ge
\frac{1}{T} \sum_{i=1}^{T}  \frac{1}{2m} \sum_{x \in C} \mathds{1}_{[A(S^i_j)(x) \neq f_i(x)]} \ge \\
&\ge \frac{1}{2p} \sum_{r=1}^p \frac{1}{T} \sum_{i=1}^{T}  \mathds{1}_{[A(S^i_j)(v_r) \neq f_i(v_r)]} \ge \\
&\ge \frac{1}{2} min_{r} \frac{1}{T} \sum_{i=1}^{T}  \mathds{1}_{[A(S^i_j)(v_r) \neq f_i(v_r)]}
\end{align*}


Como dado un $v_r$ cualquiera, $v_r \notin S_j$, y existen $f_i, f_{i'}$ que se diferencian justo en el elemento $v_r$, uno coincidirá con el valor en $v_r$ de $A(S_{j}^i) = A(S_{j}^{i'}$ y otro no:

\[\frac{1}{2} \frac{1}{T} \sum_{i=1}^{T}  \mathds{1}_{[A(S^i_j)(v_r) \neq f_i(v_r)]} = \frac{1}{2} \frac{1}{T} \frac{T}{2} = \frac{1}{4}\]

Y uniendo toda esta información:

\[max_{i \in \{1,\ldots T\}} \frac{1}{k} \sum_{j=1}^{k} L_{\mathcal{D}_i} (A(S_j^i)) \ge \frac{1}{4}\]

Sea $k = argmax_{i \in \{1,\ldots T\}} \frac{1}{k} \sum_{j=1}^{k} L_{\mathcal{D}_i} (A(S_j^i))$

Si $\mathcal{D} = \mathcal{D}_k$ cumple la parte 2 del enunciado del teorema, es nuestra distribución buscada, y como función buscada en el apartado 1. podemos tomar $f=f_k$

Como $L_{\mathcal{D}} (A(\cdot))$ puede ser vista como una variable aleatoria donde $S \sim \mathcal{D}^m$ y que toma valores en $[0,1]$, tenemos que tomando $Z = 1-L_{\mathcal{D}}(A(\cdot))$, $a=\frac{7}{8}$ en el lema previo llegamos a:

\[P_{S\sim \mathcal{D}^m} \left(\frac{1}{8} \ge L_{\mathcal{D}}(A(S)) \right) \le \frac{3}{4} \cdot \frac{8}{7} = 24/28\]

donde $\mathbb{E}(Z) = \mathbb{E} (1 - L_{\mathcal{D}}(A(\cdot))) = 1 - \mathbb{E} (L_{\mathcal{D}}(A(\cdot))) \le \frac{3}{4}$

Es decir:

\[P_{S\sim \mathcal{D}^m} \left( L_{\mathcal{D}}(A(S)) \ge \frac{1}{8} \right) \ge \frac{4}{28} = \frac{1}{7}\]
\end{proof}


Como consecuencia del teorema, podemos decir que no hay un algoritmo de aprendizaje óptimo para todas las distribuciones, puesto que para una dada por el resultado del teorema, el algoritmo ERM con $H = \{f\}$ aprendería mejor.

  \chapter{Dimensión VC}
    \section{Dimensión Vapnik-Chervonenkis}
\subsection{Introducción}
La abreviaremos dimensión VC

\begin{definition}
\textbf{Restricción de $H$ a $C$}

Sea $H$ clase de hipótesis de $X$ a $\{0,1\}$, y $C=\{c_1, \ldots c_m\} \subseteq X$. 
Llamamos restricción de $H$ a $C$ al conjunto de funciones:

\[H_{C} = \{h_{|C} : h\in H\} \cong \{(h(c_1), \ldots h(c_m)): h\in H\}\]
\end{definition}


\begin{definition}
\textbf{Restricción de $X$ a $H$}

Sea $H$ clase de hipótesis de $X$ a $\{0,1\}$, y $C=\{c_1, \ldots c_m\} \subseteq X$. Llamamos restricción de $X$ por $H$ a:

\[X_{H} = \{S \subseteq X: \exists h\in H, h(S)=\{1\} \}\]
\end{definition}


\begin{definition}
\textbf{Conjunto fragmentado por otro}

Un conjunto $\mathcal{F}$ diremos que fragmenta a otro conjunto finito $C$ si se verifica que para todo subconjunto de $D \subseteq C$ existe $S\in \mathcal{F}$ con $S \cap C = D$
\end{definition}


\begin{definition}
\textbf{Conjunto fragmentado por una clase de hipótesis}

Una clase de hipótesis $H$ fragmenta un conjunto finito $C \subseteq X$ sii la restricción de $H$ a $C$ nos da todas las posibles funciones de $C$ a $\{0,1\}$. Esto es, si $|H_{C}| = 2^{|C|}$

\begin{lemma}
\textbf{Caracterización del concepto de fragmentación de $C$ por $H$}

Una clase de hipótesis $H$ fragmenta un conjunto finito $C \subseteq X$ sii $X_{H}$ fragmenta $C$
\end{lemma}

\begin{proof}
La demostración de la caracterización es trivial desde la biyección:

\[\{h_{|C} : h\in H\} \cong \{(h(c_1), \ldots h(c_m)): h\in H\}\]

con $C = \{c_1, \ldots c_m\}$
\end{proof}

Este lema nos permite trabajar indistintamente con la fragmentación de un conjunto por una clase de funciones o por la restricción del espacio por dicha clase de hipótesis.

Cuando demostrábamos el teorema de No Free Lunch, no teníamos ninguna restricción sobre la distribución que construíamos ni sobre la hipótesis que daba lugar a esa distribución, la $f$ que cumplía que tenía error nulo. Siempre que el conjunto $C$ que tomamos sea fragmentado por $H$, podremos asegurar que la $f$ que genera la distribución pertenece a la clase de funciones $H$. Formalmente:

\begin{theorem}
\textbf{Teorema de No Free Lunch revisitado}

Sea $H$ una clase de hipótesis de $H$ a $\{0,1\}$, $m < |X|/2$ el tamaño del conjunto de entrenamiento. Supongamos que existe $C\subseteq X$ de tamaño $2m$ fragmentado $H$. Sea $A$ cualquier algoritmo de aprendizaje, entonces existe una distribución $\mathcal{D}$ sobre $X \times \{0,1\}$ verificando:

\begin{enumerate}
\item Existe una función $f: X \rightarrow \{0,1\}$, $f\in H$ con $L_{\mathcal{D}}(f)=0$
\item $P_{S\sim \mathcal{D}^m} [L_{\mathcal{D}} (A(S)) \ge 1/8] \ge 1/7$
\end{enumerate}

\label{nofreelunch-v2}
\end{theorem}

Intuitivamente, si existe un conjunto $C$ fragmentado por $H$ y nuestro conjunto de entrenamiento contiene la mitad de instancias de $C$ (recordemos que la distribución que construíamos en la demostración de No Free Lunch asignaba toda la masa de probabilidad al conjunto $C$), entonces no tenemos información suficiente para etiquetar correctamente el resto de instancias (hay demasiadas posibles hipótesis que etiquetan el conjunto de entrenamiento de igual forma pero difieren en el resto de instancias).

\#+begin$_{\text{definition}}$
\textbf{Dimensión VC}

Definimos la dimensión VC de una clase de hipótesis $H$ como el tamaño máximo de los conjuntos $C \subseteq X$ verificando que son fragmentados por $H$. Si no existe máximo, decimos que $H$ tiene dimensión VC infinita. La notamos $VC(H)$
\end{definition}


Del teorema No Free Lunch revisitado deducimos:

\begin{theorem}
\textbf{Ser PAC cognoscible implica tener dimensión VC finita}

Sea $H$ clase de hipótesis con $VC(H) = \infty$. Entonces $H$ no es PAC cognoscible
\end{theorem}

\subsection{Ejemplos}

\subsubsection{Intervalos $]-\infty, a[$}

Sea $H = \{h_a = \mathds{1}_[x<a]: a\in \mathbb{R}\}$ clase de hipótesis sobre $\mathbb{R}$. 

Dado un conjunto $C=\{\alpha\}$, podemos tomar $h_{\alpha}$ y $h_{\alpha+1}$, que nos dan todos los posibles etiquetados de $C$.
Sin embargo, dado un conjunto de tamaño 2, $C=\{\alpha, \beta\}$, donde podemos suponer spg. $\alpha < \beta$. Entonces no podemos encontrar $h_b \in H$ verificando $h_b(\alpha)=0$ y $h_b(\beta) = 1$, ya que esto implicaría que $b > \beta$ y por tanto entraría en contradicción con que $h_b(\alpha) = 0$

Hemos probado $VCdim(H) = 1$.

\subsubsection{Intervalos cerrados y acotados}

Sea $H = \{h_{a,b} = \mathds{1}_[a<x<b]: a,b\in \mathbb{R}\}$ clase de hipótesis sobre $\mathbb{R}$. 

Dado un conjunto $C=\{\alpha\, \beta\}$, con $\alpha < \beta$, las hipótesis $h_{\alpha+\delta_1, \beta + \delta_2}$ con $\delta_i \in \{0,1\}$ nos dan todos los posibles etiquetados de $C$.
Sin embargo, dado un conjunto de tamaño 3, $C=\{\alpha, \beta\, \theta\}$, donde podemos suponer $\alpha < \beta < \theta$. Entonces no podemos encontrar $h_b \in H$ verificando $h_b(\alpha)=1$ y $h_b(\theta) = 1$ y $h_b(\beta) = 0$

Hemos probado $VCdim(H) = 2$.

\subsubsection{Clases finitas}

Sea $H$ una clase finita. Entonces para un conjunto $C \subseteq H$ se tiene $|H_C| \le |H|$ y por tanto el 
conjunto no puede ser fragmentado por $H$ si $|H| < 2^{|C|}$, lo que implica $VCdim(H) \le log_2(|H|)$

\subsubsection{Dimensión VC y número de parámetros}

Puede demostrarse que la dimensión VC de los clasificadores de rectángulo $H = \{h_{a,b,c,d} := \mathds{1}_{[a\le x\le b, c\le y\le d]}\}$ en $\mathbb{R}^2$, que ya mencionamos en un ejemplo en los temas anteriores es 4. Esto, unido a los ejemplos anteriores con intervalos podría hacernos conjeturar que la dimensión VC depende del número de parámetros con el que definimos los clasificadores. El siguiente ejemplo demuestra que esto es falso.

Dada la clase de clasificadores $H = \{h_{\theta}: \theta \in \mathbb{R}\}$ donde 
$h_{\theta}: X \rightarrow {0,1}$ está definida por $h_\theta (x) = \lceil 0.5 sen(\theta x) \rceil$, y 
$d\in \mathbb{N}$ arbitrario, podemos tomar $d$ puntos codificados por $x_i = 0.x_{1,j} \ldots x_{2^d,j} x^{2^d + 1,j}$ 
donde cada $x_{i}$ es una fila de la matriz $(x_{i,j})$ donde la columna $2^{d+1}$ ésima es 1, y la columna 
$i$ -ésima codifica el número $i-1$ en binario, leído de arriba a abajo. Así dado una asignación de $d$ 
etiquetas, debe codificar un número en binario entre $1$ y $2^d-1$, a saber, la columna $k$ ésima de la 
matriz. Tomamos el clasificador $h = \lceil 0.5 sen(10^k \pi x) \rceil$ que verificará que su asignación de 
etiquetas es justamente la columna $k$ ésima. El sentido de la última columna constantemente $1$ puede 
explicarse en que daremos $x_{i,1} \ldots x_{i,k}$ medias vueltas a la circunferencia unidad y recorreremos 
y una fracción $0.x_{i,k+1} \ldots x_{2^d,j} 1$ no nula de otra media vuelta a la circunferencia. Si 
$x_{i,k} = 1$ entonces $h(x_i) = 1$, y si $x_{i,k}=0$ entonces $h(x_i) = 0$. Luego $VCdim(H) = \infty$.

\subsection{Teorema fundamental de aprendizaje PAC}

\begin{theorem}
\textbf{Teorema fundamental de aprendizaje PAC}

Sea $H$ clase de hipótesis de la forma $h: X \rightarrow \{0,1\}$ y la función de pérdida 0-1. Entonces equivalen:

\begin{enumerate}
\item $H$ tiene la propiedad de convergencia uniforme.
\item $H$ es agnósticamente PAC cognoscible por cuaquier algoritmo ERM.
\item $H$ es agnósticamente PAC cognoscible.
\item $H$ es PAC cognoscible.
\item $H$ es PAC cognoscible por cualquier algoritmo ERM.
\item $VC (H) < \infty$.
\end{enumerate}
\end{theorem}


La implicación que nos falta es $6 \implies 1$. El resto de implicaciones se consiguen a partir de teoremas ya probados en temas anteriores.

Daremos una serie de lemas y definiciones previas antes de probarla.

\begin{definition}
\textbf{Función de crecimiento}

Sea $H$ una clase de hipótesis. Definimos como función de crecimiento de $H$:

\[\begin{array}{ll}
\tau_{H}: & \mathbb{N} \rightarrow \mathbb{N}\\
                    & m          \mapsto     \underset{C \subseteq X: |C|=m}{max}{|H_C|}
\end{array}\]

Esta función está bien definida puesto que fijado $m \in \mathbb{N}$ se tiene siempre que $|H_C| \le 2^C$
\end{definition}

\begin{lemma}
\textbf{Lema de Sauer-Shelah}

Sea $H$ clase de hipótesis con $VC(H) \le d < \infty$. Entonces para todo $m\in \mathbb{N}$ se tiene $\tau_{H} (m) \le \sum_{i=0}^d \binom{m}{i}$. Se deduce que si $m > d+1$ entonces $\tau_{H}(m) \le (em/d)^d$.
\end{lemma}


\begin{proof}
Empezamos probando que una clase de hipótesis $\mathcal{F}$ finita fragmenta al menos $|\mathcal{F}|$ conjuntos.

Lo hacemos por inducción sobre el tamaño de $X_\mathcal{F}$ que es un conjunto finito por ser $\mathcal{F}$ clase finita.

Si su tamaño es 1, parte al conjunto vacío.

Supuesto que se verifica la hipótesis para tamaños menores que $k-1$ y sea $|X_\mathcal{F}| = k$. Escogemos entonces $x\in X$ verificando que $x$ pertenece a algunos conjuntos de $X_\mathcal{F}$ pero no a todos (debe existir, sino tendríamos que $X_\mathcal{F}$ contiene un único conjunto).

Sean $A = \{S \subseteq X_\mathcal{F} : x\in S\}$, $A'=\{S\setminus\{x\} : S \in A\}$, $B= \{S \subseteq X_\mathcal{F} : x\not\in S\}$.

Claramente $|A| = |A'|$ y por hipótesis de inducción $A'$ fragmenta $k \ge |A|$ conjuntos, y $B$ fragmenta $|B|$ conjuntos. También es trivial ver que los conjuntos $S$ y $S\cup \{x\}$ son fragmentados por $A$ donde $S$ es un conjunto fragmentado por $A'$. Hemos probado que $A$ fragmenta $|A| + k$ conjuntos, y si un conjunto es fragmentado por $A$ y $B$ a la vez, entoces no contiene a $x$, luego es fragmentado por $A'$ y por $B$ a la vez, y podremos tener a lo sumo $k$ conjuntos de este tipo.

En definitiva hemos probado que fragmentamos $|A| + k + |B| - k = |A| + |B| = |\mathcal{F}|$ conjuntos.
Probado esto, si $\tau_{H}(m) > \sum_{i=0}^d \binom{m}{i}$ entonces $H$ debe fragmentar un conjunto de tamaño 
$d+1$ al menos, puesto que el número de subconjuntos de un conjunto finito $C$ menor que  $d+1$ es exactamente
$\sum_{i=0}^d \binom{|C|}{i}$. Luego tendríamos $VC(H) > d$
\end{proof}
  
% *************************************************************************************************************
% Parte de informática
% *************************************************************************************************************
\part{Informática}
  \chapter{Problema del desbalanceo en clasificación binaria}
    \subsection{Introducción}
 \begin{frame}\
 \begin{definition}[Instancias positivas, negativas]
 LLamamos:
 \begin{itemize}
  \item Instancias positivas a $Z^{+} = \{(x,y)\in Z: y=1\}$.
  \item Instancias negativas a $Z^{-} = \{(x,y)\in Z: y=-1\}$.
 \end{itemize}
 Una notación equivalente se aplica a $S$, conjunto de entrenamiento. Además llamamos $\ppos = P(Z^{+})$ y $\pneg = P(Z^{-})$.
 \end{definition}

 \begin{definition}[Desbalanceo entre clases]
  Decimos que un problema de clasificación binaria es desbalanceado (entre clases) si se verifica que tomando 
  $\nneg = |S^{-}|$ y $\npos = |S^{+}|$, entonces $\npos < \nneg$.
  
  Llamamos ratio de desbalanceo a $r(S)=\frac{\npos}{\nneg}$. Normalmente $r(S) \ll 1$.
 \end{definition}

 Para ciertos conjuntos de datos, donde el número de instancias positivas con respecto a las negativas es muy pequeño, un clasificador podría simplemente
 etiquetar todas las instancias como negativas, y obtener un error global pequeño, pero no acertaría en ninguna instancia
 positiva. Si los datasets empleados representaran enfermedades, por ejemplo,
 nos interesa encontrar mecanismos para disminuir el error sobre la clase positiva (la de las instancias correspondientes
 a la enfermedad).
\end{frame}

\begin{frame}\frametitle{Tipos de desbalanceo}
\begin{columns} 
 \begin{column}{0.55\textwidth}
  El desbalanceo puede ser:
  
  \begin{enumerate}[i]
   \item \textbf{Intrínseco}, cuando $\ppos < \pneg$.

   \item \textbf{Extrínseco}, si $\ppos \ge \pneg$ pero sin embargo para nuestro conjunto de entrenamiento $S$, 
   tenemos que $\npos < \nneg$.
  \end{enumerate}
  
  
  Se dice que existe desbalanceo \textit{intra clases} cuando además la clase minoritaria está repartida en varias 
  regiones no conexas, a saber $\spos \supset A_1, \ldots, A_m$ donde $|A_1| \le \ldots \le |A_m|$, con algún menor o igual estricto.
  Esto dificulta aún más la correcta clasificación de dichas instancias, porque aparecen zonas con instancias positivas 
  pobremente representadas. A estas zonas de baja representación las llamamos instancias raras.
 \end{column}
 
 \begin{column}{0.45\textwidth}
  \img{../memoria/imgs/desbalanceo.png}{1.1}
 \end{column}
\end{columns}
\end{frame}

\begin{frame}\frametitle{Aprendizaje con desbalanceo}
\par\textbf{Técnicas de aprendizaje con desbalanceo}
\begin{itemize}
 \item \textbf{\textit{Oversampling/Undersampling}}. Consiste en tomar $E$ instancias clasificadas como positivas y entrenar con 
 $S' = S\cup E$ o $S' = S\setminus E$.
 
 Tanto el \textit{oversampling} como el \textit{undersampling} se realiza de manera que $r(S') > r(S)$. Se pueden usar además
 técnicas de limpieza, para definir mejor los bordes de las zonas que conectan a ambas clases, de manera que mejoremos el error
 en la clasificación.
 
 \item \textbf{Aprendizaje \textit{cost sensitive}}. El \textit{framework} de aprendizaje \textit{cost sensitive} implica que no
 a todas las instancias le asignamos el mismo error al aprender.
\end{itemize}
\medskip

\par\textbf{Medidas de la bondad del aprendizaje}
\begin{columns}
 \begin{column}{0.5\textwidth}
  \begin{table}[H]
    \centering
    \begin{tabular}{C{2cm}|C{2cm}}
    $VP$ & $FP$\\
    \hline
    $FN$ & $VN$\\
    \end{tabular}
    \caption{Matriz de confusión}
  \end{table}
 \end{column}
 
 \begin{column}{0.5\textwidth}
  \begin{definition}[Sensibilidad]
   Se define la sensibilidad como $s = \frac{VP}{\npos}$ 
  \end{definition}
  
  Hay otras medidas como el $F$-score, el $AUC$, \ldots
 \end{column}
\end{columns}
\end{frame}


  \chapter{Algoritmos implementados}
    \section{Algoritmos RACOG y wRACOG}

\begin{algorithm}[H]
\begin{algorithmic}[1]
  \REQUIRE $S = \{x_i=(w_1^{(i)}, \ldots w_d^{(i)})\}_{i=1}^m$, instancias
  \STATE{Calcular $G' = (E',V')$ árbol de dependencia según Chow Liu}
  \STATE{Construir $G$ un grafo no dirigido desde $G'$, $E$ arcos, $r$ raíz}
  \FOR{$(u,v) \in E$}
    \STATE{Calcular $\bar{P}(w_v \mid w_u)$}
    \STATE{Calcular $P_{abs}(w_u)$}
  \ENDFOR
  \RETURN{$P(w_1, \ldots w_m)$, construida con $P_{abs}$ y $\bar{P}$}  
\end{algorithmic}
\caption{Algoritmo AproximarDistribución}
\label{alg:aproxdist}
\end{algorithm}

\begin{algorithm}[H]
\begin{algorithmic}[1]
  \REQUIRE $S = \{x_i=(w_1^{(i)}, \ldots w_d^{(i)})\}_{i=1}^m$, instancias
  \REQUIRE $P(w_1, \ldots w_m)$, distribución conjunta
  \FOR{$i=1, \ldots, m$}
    \FOR{$k=1,\ldots, d$}
      \STATE{$\bar{w}_k^{(i)} \sim P(w_k \mid \bar{w}_1^{(i)}, \ldots, \bar{w}_{k-1}^{(i)}, w_{k+1}^{(i)} \ldots, w_{d}^{(i)})$}
    \ENDFOR
  \ENDFOR
  \RETURN{$S = \{\bar{x}_i=(\bar{w}_1^{(i)}, \ldots \bar{w}_d^{(i)})\}_{i=1}^m$, conjunto de instancias\\ generado desde $S$ y $P$}
\end{algorithmic}
\caption{Algoritmo GibbsSampler}
\label{alg:gibbs}
\end{algorithm}


\begin{algorithm}[H]
\begin{algorithmic}[1]
  \REQUIRE $\spos = \{z_1=(x_1, y_1), \ldots z_m=(x_m, y_m)\}$, ejemplos positivos
  \REQUIRE $\beta$, burnin
  \REQUIRE $\alpha$, lag
  \REQUIRE $T$, número de iteraciones
  \STATE{$S = \spos_x$}
  \STATE{$P = \textrm{AproximarDistribución}(S)$}
  \STATE{$S'= \emptyset$}
  \NEWLINE
  \FOR{$t=1,\ldots, T$}
    \STATE{$S = \textrm{GibbsSampler}(S, P)$}
    \NEWLINE
    \IF{$t > \beta$ \AND $t\mod(\alpha) = 0$}
      \STATE{$S' = S' \cup S$}
    \ENDIF
  \ENDFOR
  \NEWLINE
  \RETURN{$S'$, ejemplos positivos sintéticos}    
\end{algorithmic}
\caption{Algoritmo de \textit{oversampling} RACOG}
\label{alg:racog}
\end{algorithm}


\begin{algorithm}[H]
\begin{algorithmic}[1]
  \REQUIRE $S_{train} = \{z_1=(x_1, y_1), \ldots z_m=(x_m, y_m)\}$, conjunto de \textit{train}
  \REQUIRE $S_{val} = \{z_1=(x_1, y_1), \ldots z_m=(x_m, y_m)\}$, conjunto de validación
  \REQUIRE $wrapper$, clasificador
  \REQUIRE $T$, número de iteraciones a considerar
  \REQUIRE $\alpha$, parámetro de tolerancia
  \STATE{$S = \spos_{train}$}
  \STATE{$P = \textrm{AproximarDistribución}(S)$}
  \STATE{Obetener $modelo$ con $wrapper$ y $S_{train}$}
  \STATE{Inicializar nuevas muestras $S'= \emptyset$}
  \STATE{Inicializar $\tau = (\underset{1)}{+\infty}, \ldots, \underset{T)}{+\infty})$}
  \NEWLINE
  \WHILE{Desviación estándar de $\tau \ge \alpha$}
    \STATE{$S = \textrm{GibbsSampler}(S, P)$}
    \STATE{$S_{misc} =$ instancias mal clasisicadas de $S$ clasificadas por $modelo$}
    \STATE{Actualizar nuevas instancias, $S' = S' \cup S_{misc}$}
    \STATE{Actualizar \textit{train}, $S_{train} = S_{train} \cup S_{misc}$}
    \STATE{Obetener $modelo$ con $wrapper$ y $S_{train}$}
    \STATE{Hacer $s = $ sensibilidad de la predicción de $modelo$ sobre $S_{val}$}
    \STATE{Hacer $\tau = (\tau_2, \ldots, \tau_T, s)$}
  \ENDWHILE
  \NEWLINE
  \RETURN{$S'$, ejemplos positivos sintéticos}    
\end{algorithmic}
\caption{Algoritmo de \textit{oversampling} wRACOG}
\label{alg:wracog}
\end{algorithm}



    \section{Algoritmo RWO}
El algoritmo RWO (\textit{Random Walk Oversampling}) consiste en construir instancias sintéticas inspirándonos
en el teorema central del límite, de manera que intentamos dejar invariante el límite en distribución de los
datos de entrenamiento.

\begin{theorem}
 Sean $\{W_1, \ldots, W_m\}$ un conjunto de variables aleatorias i.i.d., con $\expect(W_i) = \mu$ y 
 $Var(W_i) = \sigma^2 < \infty$. Entonces:
 
 \[\lim_{m} P\left[\frac{\sqrt{m}}{\sigma} \left(\underbrace{\frac{1}{m}\sum_{i=1}^m W_i}_{\overline{W}} - 
   \mu \right) \le z \right] = \phi(z)\]
 
 donde $\phi$ es la función de distribución de la normal $N(0,1)$.
 
 Es decir $\frac{\overline{W} - \mu}{\sigma/\sqrt{m}} \rightarrow N(0,1)$ en probabilidad.
 
 \label{th:tcl}
\end{theorem}

Supongamos que las muestras son de dimensión $d$. Fijamos una columna $j\in \{1, \ldots d\}$. Supuesto un
conjunto de instancias de la clase positiva $\{x_i=(w_1^{(i)}, \ldots, w_d^{(i)})\}_{i=1}^m$, supondremos que la
columna $j-$ésima de los datos está generada según una variable aleatoria $W_j$ de media $\mu_j$ y desviación
típica $\sigma_j < \infty$.

Sean $\mu_j', \sigma_j'$ la media y desviación típica muestrales, respectivamente. Las instancias sintéticas 
que generaremos serán de la forma $w_j'(i) = w_j^{(i)} - \frac{\sigma_j'}{\sqrt{m}} \cdot r$, con 
$r\sim N(0,1)$, para $i=1, 2, \ldots, m$. Nótese la similitud de esta fórmula con \ref{th:tcl}, con la salvedad
de que usamos los valores para la variable $W_j$ que conocemos en lugar de $\mu_j$, que es desconocida; y 
$\sigma_j'$ que es la varianza muestral no corregida, en lugar de la verdadera varianza muestral $\sigma_j$, 
que tampoco la conocemos.

\begin{theorem}
 La esperanza de la media muestral de las instancias $\{w_j^{'}(i)\}_{i=1}^m$ sintéticas es $\mu_j$. 
 La esperanza de la varianza muestral tiende en $m$ a $\sigma_j^2$.
\end{theorem}

  \begin{proof}
   Se tiene una variable aleatoria $W_j' = W_j - \frac{\sigma_j'}{\sqrt{m}} R$ donde $R\sim N(0,1)$.
  
   Para la media:
   \[
     \expect(W_j') = \expect(W_j) - \frac{\sigma_j'}{\sqrt{m}} \expect(R) = \mu_j
   \]
   
   Puesto que $\expect(W_j) = \mu_j$ y $\expect(R) = 0$. Usando linealidad de la esperanza se prueba que
   $\expect\left(\frac{1}{m} \sum_{i=1}^m w_j'(i)\right) = \mu_j$.
   
   Ahora dado $j$ arbitrario:
   \begin{align*}
     \expect \bigg(\bigg(W_j - R{\sqrt{m}} \sigma_j' - \mu_j\bigg)^2\bigg) &= 
     \expect \bigg((W_j - \mu_j)^2 \bigg) - \expect\bigg( 2 (W_j-\mu_j) \frac{r_i}{\sqrt{m}} \sigma_j' \bigg) + \\
     &+ \expect\left(\frac{R^2}{m} \sigma_j'^2 \right) = \sigma_j^2 + 0 + \frac{1}{m}\sigma_j'^2
   \end{align*}
   
   Sea $\tau_m = \frac{1}{m} \sum_{i=1}^m \left(w_j'(i) - \frac{1}{m} \sum_{i=1}^m w_j'(i)\right)^2$ la varianza muestral.
   Es conocido que $\expect(\tau_m) = \frac{m-1}{m} Var(W_j') = 
   \frac{(m-1)}{m} \left(\sigma_j^2 + \frac{1}{m} \sigma_j'^2\right)$

   Luego $\lim_{m\rightarrow + \infty} \expect(\tau_m) = \sigma_j^2$.
  \end{proof}

El algoritmo incluye una distinción para los atributos no numéricos, donde se establece una distribución uniforme
para todos los valores que posean dichos atributos, y se le asigna a la instancia generada. En pseudocódigo
obtenemos:

\begin{algorithm}[H]
\begin{algorithmic}[1]
  \REQUIRE $S = \{x_i=(w_1^{(i)}, \ldots w_d^{(i)})\}_{i=1}^m$, ejemplos positivos
  \REQUIRE $T$, número de instancias sintéticas deseado
  \STATE{Inicializar $S'= \emptyset$}
  \NEWLINE
  \FOR{Cada atributo atributo $j=1, \ldots, d$}
    \IF{El atributo $j-$ésimo es numérico}
      \STATE{Calcular la varianza $\sigma_j' = \sqrt{\frac{1}{m}\sum_{i=1}^m \left(w_j^{(i)} - \frac{\sum_{i=1}^m w_j^{(i)}}{m} \right)^2}$}
    \ENDIF
  \ENDFOR
  \NEWLINE
  \STATE{Hacer $M = \left\lceil T/m \right\rceil$}
  \FOR{$t=1, \ldots, M$}
    \FOR{$i=1,\ldots, m$}
      \FOR{$j=1, \ldots, d$}
         \IF{El atributo $j-$ésimo es numérico}
           \STATE{Escoger $r \sim N(0,1)$}
	   \STATE{$w_j = w_j^{(i)} - \frac{\sigma_j'}{\sqrt{m}} \cdot r$}
	 \ELSE
	   \STATE{Escoger $w_j$ de manera uniforme sobre $\{w_j^{(1)}, \ldots w_j^{(m)}\}$}
	 \ENDIF
      \ENDFOR
      \STATE{$S' = S' \cup \{(w_1, \ldots w_d)\}$}
    \ENDFOR
  \ENDFOR
  \NEWLINE
  \STATE{$S'=$ Escoger $T$ instancias aleatorias de entre $S'$}
  \RETURN{$S'$, ejemplos positivos sintéticos}
\end{algorithmic}
\caption{Algoritmo de \textit{oversampling} RWO}
\label{alg:rwo}
\end{algorithm}

Implementado en R, el cuerpo principal del algoritmo, omitiendo otros detalles de la implementación, queda:

\lstinputlisting[language=R, caption = Cuerpo del algoritmo RWO]{codelst/rwo.R}
    \section{Algoritmo PDFOS}

\begin{algorithm}[H]
\begin{algorithmic}[1]
  \REQUIRE $\spos = \{z_1=(x_1, y_1), \ldots z_m=(x_m, y_m)\}$, ejemplos positivos
  \REQUIRE $T$, número de instancias sintéticas deseado
  \STATE{$S = \spos_x = \{x_i=(w_1^{(i)}, \ldots w_d^{(i)})\}_{i=1}^m$}
  \STATE{$S'= \emptyset$}
  \STATE{$\sigma =$ búsqueda con GridSearch que minimice $M$}
  \STATE{Calcular $U$ la matriz de covarianza de $S$}
  \STATE{Calcular descomposición de Choleski de $U$, donde $U=R^{T} R$, y $R$ triangular superior}
  \NEWLINE
  \FOR{$i=1, \ldots, T$}
    \STATE{Escoger $x\in S$}
    \STATE{Escoger $r$ siguiendo una normal multivariante, $r \sim N^d(0,1)$}
    \STATE{$S' = S' \cup \{x + \sigma r R\}$}
  \ENDFOR
  \NEWLINE
  \RETURN{$S'$, ejemplos positivos sintéticos}
\end{algorithmic}
\caption{Algoritmo de \textit{oversampling} PDFOS}
\label{alg:pdfos}
\end{algorithm}


Se comenzó usando el siguiente algoritmo de búsqueda \textit{grid}:

\begin{algorithm}[H]
\begin{algorithmic}[1]
  \STATE{Hacer $M_{best} = \infty$}
  \FOR{$\tau \in \{0.2, 0.22, 0.24, \ldots 2\}$}
    \IF{$M(\tau) < M_{best}$}
      \STATE{$\sigma = \tau, M_{best} = M$}
    \ENDIF
  \ENDFOR
  \RETURN{$\sigma$}
\end{algorithmic}
\caption{Algoritmo de búsqueda GridSearch}
\end{algorithm}

Posteriormente se agregaron también al algoritmo de búsqueda la comprobación con los valores de parámetro propuestos por Scott
y Silverman (siendo $d$ el tamaño del espacio de atributos, y $m$ el número de instancias pertenecientes a la clase minoritaria):

\[\sigma_{Scoot} = \left(\frac{1}{m}\right)^{\frac{1}{d+4}}, \quad \sigma_{Silverman} = \left(\frac{4}{m(d+2)}\right)^{\frac{1}{d+4}}\]

    \section{Algoritmo NEATER}
El algoritmo NEATER (\textit{filteriNg of ovErsampled dAta using non cooperaTive gamE theoRy}) es un algoritmo de filtrado 
de instancias no representativas basado en teoría de juegos. Daremos a continuación unos preliminares de teoría de juegos
para poder comprender mejor el algoritmo.

\subsection{Teoría de juegos}
Sea una tupla $(P, S, f)$, donde $P=\{1, \ldots, n\}$, conjunto de jugadores. Tendremos $S_i=\{1, \ldots, k_i\}$ conjunto
de posibles estrategias para el jugador $i$-ésimo, donde $S = S_1 \times \ldots \times S_n$, y dado $s = (s_1, \ldots s_n) \in S$
arbitrario, podremos asignarle una recompensa a cada jugador que dependerá de la estrategia que ha seguido y de la estrategia del
resto de jugadores por lo que tendremos:

\[\begin{array}{rll}
   f: S &\longrightarrow& \mathbb{R}^n\\
   s &\longmapsto& (f_1(s), \ldots, f_n(s))
  \end{array}\]
  
Notaremos $s_{-i} = (s_1, \ldots, s_{i-1}, s_{i+1}, \ldots s_n)$ y $f_i(s_i, s_{-i}):= f_i(s)$.

\begin{definition}
Un equilibrio de Nash estratégico es una tupla $s = (s_1, \ldots s_n)$ que verifica $f_i(s_i, s_{-i}) \ge f(s'_{i}, s_{-i})$ 
para cualquier otra $s'\in S$.
\end{definition}

Es decir, una estrategia de Nash maximiza la recompensa para todos los jugadores.

Tendremos probabilidades para escoger estrategias para cada jugador, esto es 
$\delta_i \in \Delta_i = \{(\delta_i^{(1)}, \ldots, \delta_i^{(k_i)}) \in (R^{+})^k_i : \sum_{j=1}^{k_i} \delta_i^{(j)} = 1\}$. 
A un vector $\delta = (\delta_1, \ldots, \delta_n) \in \Delta_1 \times \ldots \Delta_n = \Delta$ lo llamaremos perfil de estrategia. 
Asociado a un perfil de estrategia $\delta$, notaremos a la recompensa total esperada para el jugador $i$-ésimo como:

\[u_i(\delta) = \sum_{(s\in S} \delta_i^{(s_1)} f_i(s)\]

A $u_i$ lo llamaremos recompensa asociada al perfil de estrategia $x$ para el jugador $i$-ésimo, y dada $\delta\in \Delta$ notaremos
$\delta_{-i} = (\delta_1, \ldots, \delta_{i-1}, \delta_{i+1}, \ldots, \delta_n)$, y $u_i(\delta_i, \delta_{-i}):= u_i(x)$.

\begin{definition}
Un equilibrio probabilístico de Nash es un perfil de estrategia $x = (\delta_1, \ldots \delta_n)$ que verifica 
$u_i(\delta_i, \delta_{-i}) \ge u_i(x'_{i}, \delta_{-i})$ para cualquier otra $x'\in \Delta$.
\end{definition}

\begin{theorem}
 Todo juego $(P,S,f)$ con $|P| < \infty$ y $|S| < \infty$ tiene un equilibrio probabilístico de Nash.
\end{theorem}

\subsection{Aplicación al problema de clasificación desbalanceada}
En nuestro caso los jugadores serán todos los posibles puntos del conjunto de entrenamiento unido a las instancias 
sintéticas $S \cup S'$. Cada jugador podrá escoger entre dos estrategias $\{0,1\}$ donde $0$ será pertenencia a la clase 
mayoritaria y $1$ pertenencia a la clase minoritaria. Habrá dos clases de jugadores, aquellos cuya estrategia ya es fija (es
decir, conocemos su clase), que serán los de $S$, donde un jugador $k$ de $S$ siempre jugará a la estrategia $0$, esto es 
$\delta_k = (0,1)$ siempre si es una instancia de la clase negativa; y jugará a la estrategia $1$, esto es 
$\delta_k = (1,0)$ siempre, si es una instancia de la clase positiva.

Consideraremos que a una instancia sólo le afecta para su recompensa asociada a una estrategia, la suya propia y la de sus
$k$ vecinos más cercanos. Así para cada instancia $x_i \in S'$, tendremos $u_i(\delta) = \sum_{j\in NN^k(x)} (x_i^T w_{ij} x_j)$ donde
$w_{ij} = g\left(d(x_i, x_j)\right)$ tal que $g$ es decreciente (esto es, a mayor distancia, menor recompensa). En nuestro
caso, hemos tomado $g(z) = \frac{1}{1+z^2}$, con $d$ la distancia euclídea.

A cada paso se actualizarán los perfiles de estrategia de la clase minoritaria, donde para cada $x_i\in S'$ se hace: 

\begin{align*}
& \delta_i(0) = (0.5, 0.5)\\
& \delta_{i,1}(n+1) = \frac{\alpha + u_i((1,0))}{\alpha + u_i(\delta(n))} \delta_{i,1}(n)\\
& \delta_{i,2}(n+1) = 1 - \delta_{i,1}(n+1)
\end{align*}

Es decir, se va premiando a cada paso de las dos estrategias posibles, aquella que está reportando más recompensa, a base de
sustraérselo a la otra estrategia. Este proceso tiene garantizada una convergencia, por teoría de \textit{dinámica de replicador},
de teoría de juegos evolutiva.

\begin{algorithm}[H]
\begin{algorithmic}[1]
  \REQUIRE $S = \{z_1 = (x_1, y_1), \ldots z_m = (x_n, y_n)\}$, dataset original
  \REQUIRE $S' = \{\bar{z}_1=(\bar{x}_1, \bar{y}_1), \ldots \bar{z}_m=(\bar{x}_m, \bar{y}_m)\}$, ejemplos positivos
  \REQUIRE $k$, número de vecinos más cercano para KNN.
  \REQUIRE $T$, número de iteraciones deseadas.
  \REQUIRE $\alpha$, factor de suavizado.
  \STATE{Inicializar $E = \emptyset$}
  \STATE{Crear una matriz $P$ de tamaño $2\times(n+m)$ donde se inicializa: \\
    \begin{itemize} 
    \item $i=1, \ldots, |S|$, entonces $P_i = \left\{\begin{array}{ll} 
                                                    (1,0) & y_i = 1 \\
                                                    (0,1) & y_i = -1
                                              \end{array}\right.$
    \item $i=|S| + 1, \ldots, |S'|$, entonces $\delta_i = (0.5,0.5)$
    \end{itemize}}
  \NEWLINE
  \FOR{$t=1, \ldots, T$}
    \FOR{$i=1, \ldots, m$}
      \STATE{Calcular recomp. total $u_i = \sum_{x_j \in NN^k(x)} g(d(x_i,x_j))\cdot \delta_i\cdot \delta_j^T$}
      \STATE{Calcular recomp. positiva $u = \sum_{x_j \in NN^k(x)} g(d(x_i,x_j))\cdot (1,0)\cdot \delta_j^T$}
      \STATE{Calcular $\alpha = \frac{\alpha + u}{\alpha + u_i}$}
      \STATE{Actualizar $\delta_i = (\alpha, 1-\alpha)$}
    \ENDFOR
  \ENDFOR
  \NEWLINE
  \FOR{$i=1, \ldots, m$}
    \IF{$\delta_{i1} > 0.5$}
      \STATE{$E = E\cup \{(\bar{x},1)\}$}
    \ENDIF
  \ENDFOR
  \NEWLINE
  \RETURN{$E\subseteq S'$, conjunto de instancias positivas filtrado}
\end{algorithmic}
\caption{Algoritmo de limpieza de instancias NEATER}
\label{alg:neater}
\end{algorithm}


\bibliography{references}
%\bibliographystyle{apacite} 
%\bibliographystyle{apalike}
\bibliographystyle{IEEEtran}

\end{document}